\section{Discussion}
\label{sec:discussion}

Out of the five naming systems we examine, we find none so far that present an 
entirely intractable problem for defenders. For a naming system to present a  
threat, it must be both easily usable by malware authors and 
popular enough that blocking its bootstrap nodes, or blocking 
access to it entirely, will cause significant collateral damage to licit users.
For a system to be widely adopted by licit users, it must have three necessary 
characteristics. 

First, the system's name management must be as 
easy or easier than name management on traditional DNS domains. Users must not 
be required to write 
code themselves to interact with smart contracts, as is currently the case with 
each of the systems we 
study if the user does not use a custodial wallet. Users also 
must not be required to run a blockchain node in order to 
manage their names, as 
Handshake currently requires to the best of our knowledge.

Second, the transactions that are required to register and update names must be 
affordable. 
Transactions on Ethereum, in our experience, cost anywhere between \$60 and 
\$140 during the course of 
our experiments, although we discovered that we were attempting to make 
transactions during periods of 
high network congestion and fees were unusually high. Even transaction fees as 
low as ten dollars per 
transaction are far less affordable than transaction fees on naming-specific 
chains, which can be as 
low as a few cents. This dynamic may make ordinary users more likely to embrace 
naming systems built 
on naming-specific chains, rather than general-purpose chains. However, 
general-purpose chains may be 
better known, and therefore more likely to be trusted by users even if 
transaction fees are higher 
than on naming-specific chains. A trade-off may therefore exist between 
affordability and perceived 
trustworthiness and name recognition.  

Third, licit users are unlikely to embrace any naming system that does not have 
widespread browser 
adoption. Browser adoption is hindered by naming systems' lack of coordination, 
which currently leads 
to name collisions: for example, the alt-TLDs 
\texttt{.wallet}, 
\texttt{.coin}, and \texttt{.x} 
are currently used by multiple blockchain naming systems. Some newly created 
ICANN TLDs also collide 
with Handshake TLDs, such as \texttt{.music}. Naming collisions present a 
barrier to browser adoption 
because the browser would either have to enforce some sort of precedence for 
systems that include 
colliding names, or users would have to choose which naming system to use for 
each name with 
collisions. Either option will confuse and frustrate users who are unfamiliar 
with the concepts of 
namespaces. So far, only browsers that focus on privacy as one of their primary 
features have chosen 
to resolve alternate naming systems, and none have chosen to resolve systems 
that might collide with 
either each other or ICANN TLDs. Until browsers can resolve an alternate naming 
system natively, users 
are unlikely to adopt that naming system.

We conclude that the higher ease of use of purchasing, 
modifying, and resolving traditional DNS 
domains is a very high barrier for blockchain-based naming 
systems to overcome. As long as blockchain naming systems are 
not widely adopted, we predict they will not become entirely 
intractable problems for defenders.