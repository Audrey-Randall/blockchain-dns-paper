\documentclass[conference]{IEEEtran}
%\documentclass[10pt,sigconf,letterpaper]{acmart}

%\renewcommand\footnotetextcopyrightpermission[1]{} % 
%removes footnote with conference info
%\setcopyright{none}
%\settopmatter{printacmref=false, printccs=false, 
%printfolios=false}
%\pagestyle{plain}


\usepackage{color}
%\usepackage[nolist]{acronym}
%\usepackage{amsmath,amssymb}
\usepackage{pifont}
%\usepackage{enumitem}
\usepackage{booktabs}
\usepackage{url}
\usepackage{xcolor}
%\usepackage{times}  % Times fonts look better
\usepackage{array}  % Extended styles for tables
\usepackage{textcomp}
\usepackage{cite}
\usepackage{amsmath,amssymb,amsfonts}
\usepackage{algorithmic}
\usepackage{graphicx}
\usepackage{textcomp}
\usepackage{caption} 
\captionsetup[table]{skip=10pt}
\def\BibTeX{{\rm B\kern-.05em{\sc i\kern-.025em b}\kern-.08em
		T\kern-.1667em\lower.7ex\hbox{E}\kern-.125emX}}

% \iffalse
\iftrue
\newcommand{\randall}{\ding{110}\ding{43}\textcolor{magenta}}
\newcommand{\pes}{\ding{110}\ding{43}\textcolor{blue}}
\newcommand{\geoff}[1]{\ding{110}\ding{43}\textcolor{violet}{#1}}
\else
\newcommand{\randall}{}
\newcommand{\geoff}[1]{}
\fi

%\DeclareFontFamily{\encodingdefault}{\ttdefault}{\hyphenchar\font=`\-}

% \setcopyright{acmcopyright}
% \copyrightyear{2018}
% \acmYear{2018}
% \acmDOI{10.1145/1122445.1122456}

%\copyrightyear{2022}
%\acmYear{2022}
%\acmConference[IMC '22]{Internet Measurement 
%Conference}{October 25--27, 2022}{Nice, France}
%\acmConference{}{}{}
%\acmBooktitle{Internet Measurement Conference (IMC '22), 
%October 25--27, 2022, Nice, France}
%\acmPrice{TBA}

\begin{document}

\author{Anonymous Authors}
%\author{Audrey Randall}
%\email{aurandal@eng.ucsd.edu}
%\affiliation{UC San Diego}
%\author{Peter Snyder}
%\email{pes@brave.com}
%\affiliation{Brave Software}
%\author{Alisha Ukani}
%\email{aukani@ucsd.edu}
%\affiliation{UC San Diego}
%\author{Alex Snoeren}
%\email{snoeren@cs.ucsd.edu}
%\affiliation{UC San Diego}
%\author{Geoffrey M.\ Voelker}
%\email{voelker@cs.ucsd.edu}
%\affiliation{UC San Diego}
%\author{Stefan Savage}
%\email{savage@cs.ucsd.edu}
%\affiliation{UC San Diego}
%\author{Aaron Schulman}
%\email{schulman@cs.ucsd.edu}
%\affiliation{UC San Diego}
%\author{Wes Hardaker}

\title{The Challenges of Blockchain-Based Naming Systems for Malware Defenders}
%\renewcommand{\shortauthors}{A. Randall \textit{et al.}}

%\runningtitle{Article title}

%\subtitle{...}

\maketitle
\pagestyle{plain}

\begin{abstract}
	
%Malware operators require the flexibility to change the 
%location of their C2 servers in response to takedowns, 
%without 
%losing control of infected hosts. This flexibility requires 
%a naming layer.
%
%Malware on infected hosts uses a variety of means to contact 
%its command and control servers. 
%
%To maintain control over infected hosts, malware operators 
%use a variety of means to locate their C2 servers. All of 
%these methods involve a naming layer, to allow for 
%flexibility in changing the C2 server's location in the 
%event 
%of a takedown.

%Malware requires a naming layer to contact its command 
%and control (C2) servers, because infected hosts must have 
%the flexibility to change where they expect the C2 server to 
%be. Until recently, DNS domains were the obvious choice for 
%implementing this naming layer, 
%but DNS domains may be taken down by the registrar that 
%sold them. 

Successful malware campaigns often rely on the ability of 
infected hosts to locate and contact C2 servers. Until 
recently, malware campaigns generally used DNS domains for 
this purpose, but DNS domains may be taken down by the 
registrar that sold them. In response to this threat, malware 
operators have begun using \emph{blockchain-based naming 
systems} to store C2 server names. Blockchain naming systems 
are a 
threat for malware defenders because they are not subject 
to a centralized authority, such as a registrar, that can 
take down abused domains, either voluntarily or under 
legal pressure. In fact, blockchains are robust against a 
variety of interventions that work on DNS domains, which is 
bad news for defenders. 

We analyze the ecosystem of blockchain naming systems and 
identify new locations for defenders to stage interventions 
against malware. In particular, we find that 
malware is obligated to use centralized or 
semi-centralized infrastructure to connect to 
blockchain naming systems and modify the records stored 
within. We also present a study of how blockchain naming 
systems are currently abused by malware operators, and 
discuss the factors that would cause a blockchain naming 
system to become an unstoppable threat. We conclude that
existing blockchain naming systems still provide 
opportunities for defenders to prevent malware from 
contacting its C2 servers.
\end{abstract}

\section{Introduction}

%\begin{itemize}
%	\item Malware, particularly botnets (any other types? ransomware?) 
%	uses DNS to reach CNC 
%	servers. 
%	\begin{itemize}
	%		\item Malware needs a naming layer because of the 
	%		sunk infrastructure cost: 
	%		any malware already deployed that uses an IP that gets 
	%blocked/taken 
	%		over is now useless. Malware authors want a record they can change.
	%		\item Naming layer needs to be hard to block at both 
	%		the request level and the system level, so that 
	%		already-distributed malware doesn't lose access and 
	%		become useless. 
	%	\end{itemize}
%	\item DNS is decentralized in that there are many resolvers, but 
%	centralized in that there are centralized authorities. Defenders can serve 
%	legal takedown notices to those centralized authorities to block malware's 
%	access to its CNC servers.
%	\item Pivot: Malware is starting to use a truly decentralized naming 
%	system, ``blockchain DNS.'' This creates several challenges for defenders:
%	\begin{itemize}
	%		\item No central authority is capable of enacting domain-level 
	%takedown 
	%		orders
	%		\item For large chains, there is enough legitimate content that 
	%		blocking access to the whole chain is infeasible
	%		\item Transactions on large chains cost a lot of money. This limits 
	%		defenders' abilities to stage interventions like pre-registering 
	%all 
	%		domains listed in a malware's DGA.
	%	\end{itemize}
%	\item We study the ecosystem and point out several places where 
%	interventions are still possible, and discuss the challenges to defenders 
%	and occasional advantages they gain when malware uses decentralized naming 
%	such as blockchain DNS.
%\end{itemize}

Malware that is distributed across multiple machines needs a way to distribute 
commands, upload stolen 
data, and coordinate between infected hosts. Most malware, such as botnets or 
ransomware, uses a 
central command and control (C2) server for this task. However, as a single 
point of failure, a 
central C2 server presents an obvious weak link for defenders to 
target~\cite{kesari_deterring_2017}. 
Malware authors must therefore be able to easily relocate and replace a C2 
server after a defender 
takedown. All previously infected hosts must be able to find the new server at 
its new address, 
without outside coordination --- if they cannot, they become useless. Malware 
authors avoid 
this ``sunk cost'' problem by providing a layer of indirection --- a naming 
layer --- instead of 
hard-coding a fixed C2 server address directly into deployed malware. This 
naming layer must be 
resilient to takedown efforts.

Until recently, the naming layer used most frequently by malware was ordinary 
DNS, which is rarely 
blocked at the protocol level, universally supported, and easy to configure. 
Malware authors use 
various strategies, such as DGAs (domain generation 
algorithms), 
to cycle through domains and complicate defense efforts. 
However, 
DNS domains are subject 
to central authorities (registrars), who may be compelled to 
seize or deny access to abused 
domains. Malware 
authors have recently come up with an innovative solution to this risk: they 
have started to use 
\emph{decentralized naming systems}, in particular naming systems built on 
blockchains. 

Blockchain naming systems present several potential 
challenges for 
defenders. First, because they 
have no central authority to carry out legal takedown 
requests, they are immune to one of the most effective tools 
in malware defenders' arsenals. Second, some blockchain 
naming systems 
have high transaction costs to register and manage domains, 
which renders some existing defense 
strategies ineffective. For example, registering all domains 
that a DGA can generate 
is impractical in expensive blockchain naming systems. 
However, blockchain naming systems 
present challenges to malware authors as well, such as how to stealthily access 
the system.

We study the blockchain naming ecosystem and point out 
several places where defender interventions are still 
possible. In particular, we argue that infected hosts must 
pass through infrastructure that is at least 
partially centralized to access any blockchain naming system, 
and these centralized or partially centralized 
``chokepoints'' still present viable locations to stage 
interventions. We also study five existing decentralized, 
blockchain-based naming systems, and discuss the challenges 
and advantages that each present to malware authors and 
defenders. We conclude that which blockchain 
naming systems present a significant threat, defenders still 
have viable options for enacting C2 takedowns.

The remainder of this paper is organized as follows. 
Section~\ref{sec:background} describes how malware currently 
uses DNS domains to contact C2 servers, how defenders 
stage interventions against these domains, and how blockchain 
naming systems render these interventions ineffective. 
Section~\ref{sec:naming_systems} gives an overview of the 
five most widely adopted blockchain naming systems. 
Sections~\ref{sec:accessing_records} describes where defenders 
can intervene in malware's attempts to access naming systems 
and control naming records. Section~\ref{sec:case_studies} 
lists three case studies of interventions against blockchain names and their 
lessons for defenders. Section~\ref{sec:b-root} details 
measurements of how blockchain naming systems are used by 
malware today. Section~\ref{sec:discussion} discusses the 
theoretical requirements for a naming system to be a threat, 
Section~\ref{sec:related} presents related work, and 
Section~\ref{sec:conclusion} concludes.
\section{Background}

From a malware author's perspective, an ideal naming system for C2 
addresses must be uncensorable at both the request level and the system level. 
To be 
uncensorable at the request level, there should be no central authority that 
has the ability to enforce a legal takedown notice for an individual record. To 
be uncensorable at the system level, the system should be valuable enough to 
licit actors that authorities cannot block access to it entirely without 
causing significant collateral damage to benign users.

To some extent, a trade-off exists between these features. On 
one side of the 
spectrum, protocols like Tor provide high resistance to request-level 
censorship, but they stand out in network traffic, allowing systems like 
IDSes to detect the 
malware's presence. For example, enterprise networks often block Tor 
entirely under the assumption 
that none of their employees will use it for any legitimate purpose.
%but do not have enough licit traffic to prevent authorities from 
%blocking access to the system entirely. \randall{Cite that paper that   
	%said when you take away the malware, what's left is 80\% CSAM, and find 
	%other 
	%citations that say defenders block Tor.}
On the other side of 
the scale, malware has repurposed ubiquitous, benign systems 
such as social media to store C2 addresses. Defenders do not 
want to impose blanket bans on applications accessing social 
media URLs, but social media companies such as Facebook and 
Twitter have the capability, motivation, and legal obligation to enforce 
legal takedown requests on individual posts. Malware authors want to use 
systems that are balanced 
between being difficult to block at the record level and difficult to detect 
and block at the 
system level.

\subsection{DNS-based domain names}

Traditional DNS has high system-level resistance to censorship but low 
request-level resistance. 
Since nearly every device and application on the Internet requires DNS, 
enterprises and firewalls 
almost never block it entirely. However, while DNS is a decentralized 
system in terms of how its 
components are replicated across the globe, it is not decentralized in terms 
of authority. The 
registrar that sold a domain can be compelled to ``delete'' that domain or 
prevent it from being 
updated or transferred. 

The usual process for enacting a legal takedown in the United States works as 
follows~\cite{knight_domain_2015}. Upon 
identifying a domain that is being used as a C2 center, a law enforcement 
entity may choose to make 
an Acceptable Use Policy (AUP) complaint or may immediately seek a court order 
compelling the 
registrar to take down the domain. Some registrars cooperate with AUC 
complaints without legal 
intervention, but others do not. If the registrar does not respond to the 
AUC complaint and take 
down the domain, law enforcement may move on to using a court order. The 
court order commonly 
prevents the domain from being updated or transferred rather than deleting it 
entirely, because if 
the domain is deleted, it can be re-registered by the malware authors. The 
court order also 
specifies whether the domain should continue to resolve, resolve to a new IP 
address specified by 
the order, or stop resolving altogether. 

Court orders may be obtained by civil parties as well. The most common method 
is for a company to 
apply for a temporary restraining order (TRO), which orders the perpetrators 
of the offending 
activity to cease and desist and requires any intermediaries that provide 
services to the 
perpetrators to cease providing those services~\cite{kesari_deterring_2017}. 
The latter requirement 
is what allows companies to require registrars to ``sinkhole'' C2 
domains. 

Legal domain takedown orders are a critical tool for defenders to disable 
botnets. For example, 
Microsoft obtained various court orders allowing it to sinkhole the C2 servers 
of the botnet 
Citadel in 2013. Microsoft successfully argued that Citadel caused the Windows 
operating system to 
behave maliciously and frustrate users while still bearing Microsoft 
trademarks, as well as forcing 
Microsoft to spend money on security features to combat 
it~\cite{lerner_microsoft_2014}. The 
Coreflood, Kelihos, and Rustock botnets were each disabled using legal 
takedown orders obtained by 
Microsoft, Pfizer (which claimed to suffer reputational harm because Rustock 
sent spam emails for 
counterfeit pharmaceuticals), and the Department of 
Justice~\cite{kesari_deterring_2017}. These 
takedowns are possible because in each case, some centralized entity had 
control of the C2 servers' 
domain names, and thus had the capability to take them down when legally 
compelled to. Malware 
authors are thus incentivized to find naming systems that are not vulnerable to 
legal takedowns.  

\subsection{Blockchain-based domain names}

Blockchain-based naming systems present a potential threat 
because they claim to be immune to takedowns. No central authority 
controls blockchain domains in the same way that registrars 
control traditional DNS names. In general, it is not possible to modify or 
delete a record on a blockchain without controlling the record's private 
key. Once a domain has 
been registered, its ownership is passed to the purchaser, after which 
point even the company that 
sold it cannot modify it. The record is also stored immutably on the 
blockchain for as long as the 
chain exists, even if the owner later modifies or deletes it, because 
historical recods are part of 
the blockchain and cannot be erased. Blockchain-based naming systems 
therefore have high resistance 
to request-level censorship.

Furthermore, some blockchains are popular enough to be challenging to block 
without 
harming licit users. Blockchains such as Bitcoin and Ethereum have recently 
skyrocketed in popularity as investors became interested in cryptocurrency as 
an asset class. Because these blockchains are no longer niche systems used 
only by a few 
technologically-savvy users, blocking access 
to them entirely may be difficult without angering a large number of 
legitimate users. As far as we are aware, cryptocurrencies and the blockchains 
they rely on are the first examples of strongly censorship-resistant systems 
that have gained a substantial community of legitimate users around the world.

Blockchain is therefore an attractive option for malware authors to use as a 
naming 
layer for C2 server addresses. In fact, some malware is already exploiting 
these tools.
BazarLoader uses the Emercoin ``.bazar'' TLD to record the domains of its C2 
servers~\cite{brandt_bazarloader_2021}. The Namecoin TLD 
``.bit'' is used by the
Necurs botnet~\cite{dgas_of_necurs}, the Chthonic banking 
trojan~\cite{malware_traffic_analysis_2016}, Smoke Loader/Dofoil, 
Backdoor.Teamviewer, Shifu, 
and TinyNuke~\cite{abusech_2017, mackie_cryptodns_2018}. Cerber ransomware 
has 
even used blockchain wallet addresses as names for its C2 
servers~\cite{pletinckx_malware_2018}.

However, we find that blockchain domains are not as invincible as they 
claim. While sinkholing 
blockchain domains through their registrars is not possible, blockchain 
domains have two different 
weaknesses that DNS does not have: accessing the system and modifying records.

Blockchain domains are not accessible through traditional DNS. Instead, 
malware 
must find a way to contact a member of the blockchain to resolve its chosen 
domain. Malware authors 
can implement this in two ways: they can either use a centralized proxy with a 
known domain to gain 
access to the blockchain, or they can act as first-class participants of the 
blockchain themselves, 
and participate in the peer-to-peer discovery protocol. Both approaches 
present opportunities for 
defenders to intervene. 

Modifying records on blockchains also poses new challenges for both defenders 
and malware authors. 
Modifications are usually more expensive (in terms of money) than in DNS, 
making fast flux or DGA 
impractical on some chains. Similarly, if a defender wishes to register 
all domains that a DGA 
might produce, this becomes expensive on some chains.

In the remainder of this work, we detail the specific challenges posed by the 
blockchain DNS 
ecosystem to malware authors and defenders, as well as the pieces of the 
ecosystem where 
effective interventions might be staged.
\section{Overview of Blockchain Naming Systems}
\label{sec:naming_systems}

\begin{table}
	\begin{tabular}{r l l}
		\toprule
		Name system & TLDs & Proxies \\
		\midrule
		Namecoin & .bit & BDNS (defunct), \\
		& & PeerName \\
		Emercoin & .lib, &  friGate, \\
		& .bazar, &  PeerName,\\
		& .coin, & OpenNIC \\
		& .emc & \\
		Handshake & \emph{any string} & hns.to, \\
		& & NextDNS, \\
		& & HDNS.io,\\
		& & BobWallet extension, \\
		& & LinkFrame extension \\
		ENS & .eth & eth.link, \\
		& & eth.limo \\
		Unstoppable & .crypto, & Brave browser, \\
		& .blockchain, & Opera browser, \\
		& .bitcoin, & Unstoppable browser, \\
		& .coin, & Unstoppable extension, \\
		& .nft, & Infura\\
		& .wallet, & \\
		& .888, & \\
		& .dao, & \\
		& .x, & \\
		& .zil & \\
		\bottomrule
	\end{tabular}
	\caption{Non-exhaustive selection of proxies, browsers, 
		and extensions 
		that can be used to access blockchain-based naming 
		systems.}
		%\randall{maybe should have a citation for each of these}}
	\label{tab:proxies_and_tlds}
\end{table}

In this section we present an overview of five blockchain-based naming 
systems, to provide background on how such systems work in detail. We select 
these systems based on their apparent popularity, as well as 
prior reports and literature that indicate some of them have already been 
abused by malware. These naming systems fall into two categories: systems built 
on naming-specific blockchains like Namecoin and Emercoin, whose purpose is 
primarily to store names and records, and systems built on general-purpose 
blockchains such as Ethereum, that are designed for purposes beyond naming 
systems. These systems also fall into two ``generations:'' Namecoin and 
Emercoin have existed since 2011 and 2013 
respectively~\cite{namecoin, emercoin}, while the Ethereum 
Name Service (ENS), Handshake, and Unstoppable 
Domains are more recent inventions (2017, 2018, and 2019, 
respectively~\cite{ens_website, handshake_website, 
first_ud_txn}). 

All of the blockchain-based naming systems we study differentiate their names 
from DNS domains by creating alternate top level domains, 
which we refer to as \emph{alt-TLDs} for brevity. A 
summary of the alt-TLDs used by each naming system is presented in 
Table~\ref{tab:proxies_and_tlds}. Handshake names are slightly different, since 
the goal of the Handshake project is to replace the DNS root zone and make any 
alt-TLD available for purchase --- see Section~\ref{sec:handshake} for more 
details.

%Although each of these systems are blockchain-based naming 
%systems, they differ in several key ways, which are relevant to how defenders 
%must treat them when they are abused by malware authors. We now summarize each 
%system in more detail.

\subsection{Naming-Specific Blockchains}

We study three naming systems that are built on 
naming-specific and eponymous blockchains: 
Namecoin, Emercoin, and Handshake. All of these blockchains 
are primarily designed to support their naming systems,  
rather than to create new cryptocurrencies or support arbitrary 
blockchain-native programs (``smart contracts''). Because 
these 
blockchains have such specific purposes, they differ 
from blockchains like Ethereum in two ways: they have fewer 
participants and users, and their transaction fees are much 
less expensive. Both properties have implications for 
defenders --- see Section~\ref{sec:accessing_records} 
for more details. 

\subsubsection{Namecoin and Emercoin}

Namecoin and Emercoin, which are both modified copies of Bitcoin, are 
the oldest blockchain-based naming systems. Both were 
intended as additions to traditional DNS: users registered domains that 
resolved to IP addresses, using records very similar to DNS 
records. Unfortunately, Namecoin and Emercoin have been subject to a large 
amount of abuse. Four years after Namecoin's launch, Kalodner et al. found that 
only 28 of the 120,000 domains registered in Namecoin had meaningful web 
content, and most domain registrations appeared to be 
squatting~\cite{kalodner_namecoin_2015}. 
In 2021, Casino et al. collected all of the IP addresses stored by Emercoin 
and Namecoin records, and submitted them to threat intelligence services 
including VirusTotal, Hybrid Analysis, Abuse.ch, and Pydnsbl (an aggregator of 
blocklists). They found that over 50\% of the IPs in Namecoin and Emercoin 
records had been flagged as malicious by at least one threat intelligence 
service~\cite{casino_unearthing_2021}. Furthermore, Casino et al. used a 
``poisoning'' approach to find IP addresses associated with malicious IPs, 
either because the two addresses were stored in the same 
wallet, a name resolved to both addresses at different times, 
or the same email was recorded in their records. This ``poisoning'' approach 
revealed that the vast majority of IP addresses in Namecoin and Emercoin 
records are connected in some way to malicious 
IPs~\cite{casino_unearthing_2021}. 
%While blocklists may 
%contain false positives, analysis of malware binaries has 
%also revealed dozens of hard-coded Emercoin and Namecoin 
%domains that are used to contact C2 servers, as well as DGAs 
%that generate Namecoin domains for the same 
%purpose~\cite{citations_needed}. 
Our own findings support the conclusion that these naming 
systems are still rife with abuse (Section~\ref{sec:b-root}). 

%The names registered in Namecoin and Emercoin have been well studied in 
%prior work, so we did not conduct our own study of them, 
%unlike the other three naming systems we study.


\subsubsection{Handshake}
\label{sec:handshake}
\begin{table}
	\begin{tabular}{lr}
		\toprule
		Record & Names with Record \\
		\midrule
		Default NS and GLUE4 records & 102,386 \\
		\hspace*{0.2in} No A records & 102,285\\
		\hspace*{0.2in} A 44.235.163.135 & 94 \\
		\hspace*{0.2in} A 52.43.158.89 & 4 \\
		\hspace*{0.2in} A 144.91.114.245 & 2 \\
		\hspace*{0.2in} A 1.1.1.1 & 1 \\
		Invalid name & 98,068 \\
		No record (null) & 845 \\
		TXT record & 138 \\
		\hspace*{0.2in} ``hello fx-wallet'' & 110 \\
		\hspace*{0.2in} Other & 28 \\
		Non-default NS record & 32 \\
		Non-default GLUE4 record & 11 \\
		Distributed storage address & 7 \\
		\midrule
		Total unique names & 201,458 \\
		Total records & 201,487 \\
		\bottomrule
	\end{tabular}
	\caption{Record types in the Handshake namespace.}
	\label{tab:handshake_records}
\end{table}

Handshake is a blockchain-based naming system that aims to replace the root DNS 
zone. It offers its users the ability to purchase nearly any 
string to 
use as an alt-TLD. Rather than selling second-level domains 
itself, Handshake allows its users 
to act as registrars who can sell their own domains. 
Handshake records are designed to store the NS records of traditional 
authoritative nameservers, rather than to replace DNS A, AAAA, or similar 
records. Handshake 
also allows users to 
store TXT records, which can contain the addresses for decentralized web 
hosting systems like Skynet~\cite{skynet} or IPFS~\cite{ipfs}. Malware 
operators could 
potentially use 
Handshake as a naming system to find C2 content stored in 
these distributed storage systems. Additionally, Handshake 
advertises themselves as ``the only naming blockchain with a lightweight 
recursive DNS resolver, which you can easily embed into 
browsers, apps, and devices''~\cite{namebase_access_handshake}. 
This lightweight resolver may be attractive to malware 
operators because it is 
small enough to be part of a malware payload.

%\randall{Summarize the .music debate and cite it somewhere.}

To get a sense for how people use Handshake, we collected a 
sample of approximately 201,000 recently registered Handshake 
names by scraping a Handshake block 
explorer.\footnote{https://e.hnsfans.com/names} 
We attempted to scrape these names 
directly from the Handshake blockchain, but were 
unsuccessful because the RPC 
provided by the Handshake client to collect registered names 
from a Handshake node is no 
longer functional~\cite{hns_rpc_too_big}. 
Table~\ref{tab:handshake_records} 
summarizes our findings. At the moment, Handshake names appear to be 
overwhelmingly utilized as speculative assets. Only 0.14\% of 
names in our sample had NS records that differed from the 
default. Of the names that kept the default nameserver and 
glue records, only 101 (0.05\%) eventually resolved to A records, 94 of 
which were for the same IP address (a nameserver run by Namebase). Nearly half 
of registered Handshake domains in our sample cannot be 
resolved by the HNS client, since they contain illegal characters like emojis 
or are solely composed of numbers: these names are nevertheless allowed to be 
created on the Handshake blockchain. We concluded that the Handshake system has 
not yet seen significant adoption by either licit users or malicious 
actors.

\subsection{Naming Systems on General Purpose Blockchains}

Two naming systems based on the Ethereum blockchain have 
arrived since 2017: the Ethereum Name 
Service (ENS) and Unstoppable Domains. These naming systems 
are possible because of Ethereum's innovation in the 
blockchain space: \emph{smart contracts.} Smart contracts 
are code that is embedded into the Ethereum 
blockchain. Any machine that runs an Ethereum ``full node'' 
can execute any 
smart contract. Each contract is identified by a 20-byte 
address, and makes its functions available through its 
Application Binary Interface (ABI). 
Thus, asking a smart contract to execute one of its functions 
is similar to 
making an RPC call, except that instead of one machine executing the code, 
every machine that receives the transaction must do so. 

Smart contracts can be used to implement key-value stores, which means they are 
well suited to act as naming 
systems. For example, in a simplified system, a user might wish to set the name 
``foo.crypto'' to resolve to the IP address 1.2.3.4. The user would create an 
Ethereum transaction that asks the key-value store's smart 
contract 
to call its ``set record'' function, with ``foo.crypto'' and 
``1.2.3.4'' as function inputs. 
This transaction is then broadcast to the Ethereum network, 
and every Ethereum 
node that receives it updates its own copy of the key-value store to include 
the new record. Reading from the key-value store works similarly to writing to 
it: any Ethereum node can return a correct response. 
%Implication: taking down 
%the host of a name on Ethereum is not possible without taking down every 
%Ethereum node. In contrast to IPFS, where only one machine stores the content 
%you're looking for, so taking it down might be a lot easier.
Notably, any transaction that causes a write costs a ``gas 
fee'' of Ethereum cryptocurrency. Gas fees are 
dependent on network 
congestion as well as other factors: they incentivize 
Ethereum node operators to execute smart contract code, which 
uses computing resources. In contrast, reading a 
smart contract's data does not cost a gas fee and does not 
create a transaction.

Interestingly, ENS and Unstoppable Domains are structured 
like DNS, but they are not necessarily being used as DNS 
replacements. The language and structure of both systems' 
smart 
contracts implies that they were 
modeled after DNS: for example, both systems use certain 
smart contracts 
as registries, and ENS even uses others as resolvers and 
registrars. However, users are primarily using these systems 
to map human-readable names to \emph{cryptocurrency wallet 
addresses} 
instead of IP addresses. While users can still store IP 
addresses, traditional domains, TXT records, or distributed 
storage system (DS) addresses, 
very few choose to do so. This may imply that C2 records containing IP 
addresses or DNS domains will stand out and be easier for defenders to detect.

\subsubsection{ENS}

\begin{table}
	\begin{tabular}{lrr}
		\toprule
		Resolver Name & Txns Setting Resolver & Address \\
		\midrule 
		Public Resolver 2 & 33,304 & \texttt{0x4976fb...} \\
		Public Resolver 1 & 2,736 & \texttt{0xDaaF96...} \\
		OpenSea ENS resolver & 482 & \texttt{0x9C4e9C...} \\
		ENS Old Public Resolver 2 & 440	& \texttt{0x226159...} \\
		Umbra: Stealth Resolver & 409 & \texttt{0xB37671...} \\
		\textit{unnamed PublicResolver} & 126 & \texttt{0xD3ddcC...} \\
		\textit{unnamed PublicResolver} & 103 & \texttt{0x5FfC01...} \\
		ENS Old Public Resolver 1 & 29 & \texttt{0x1da022...} \\
		\bottomrule
	\end{tabular}
	\caption{The ENS resolvers from which we collected a 
	sample of names and records.}
	\label{tab:ens_resolvers}
\end{table}

\begin{figure}[t]
	\centering
	\includegraphics[width=3in]{figs/ens_names.pdf}
	\caption{Records stored by ENS names. *key within ``text'' record}
	\label{fig:ens_records}
\end{figure}

ENS names are registered (and resolved) in two steps involving two different 
smart contracts. First, a name must be registered using the ``ENS Registrar 
Controller'' smart contract, which accepts the human-readable name and the 
address of a contract to use as a ``resolver.'' Second, the 
resolver
contract must be updated with the name's records. To complicate matters, names 
are not handled in their human-readable forms after they are registered: 
instead, they are referred to by their keccak256 hash.  
Furthermore, the ENS 
Registrar Controller contract allows users to specify a 
hash instead of a human-readable name, without ever performing a transaction 
that reveals the name itself. Therefore, to enumerate most of the names in ENS, 
we had to parse all of the transactions in the ENS Registrar 
Controller 
contract that recorded new hashes of names. We then queried the associated 
resolvers to discover the human-readable names. At the time of writing, at 
least 504 smart contracts had been set as resolvers for at least one name hash.
We chose to take a sample of names from the eight resolver contracts that were 
set by the most names as their default resolver. The distribution of resolvers 
is long-tailed: the majority of resolvers resolve only a few names, while the 
eight most popular resolvers resolve the majority of names. We excluded 
addresses that were set as resolvers by many names but did not implement the 
ENS resolver specification, under the assumption that these were mistakes. 
Such misconfigured resolver addresses include the null 
address, 0x0, as well as unrelated smart contracts used by 
the ENS ecosystem. The resolvers we 
chose are detailed in Table~\ref{tab:ens_resolvers}. This 
approach yielded a 
sample of 667,369 ENS names that were registered through the 
ENS Registrar Controller contract. Prior work has found that 
even after collecting all transactions from 
the ENS Registrar Controller and its historical 
predecessors, some hashes appear in the system but have never been seen to 
resolve to 
names~\cite{xia_ens_2022}. It is unclear how these 
hashes came to exist, so we note that our sample does not 
contain all of the names in ENS, just the majority.

%To 	fully resolve a 
%name, a user must first query the ENS Registrar Controller 
%to determine the 
%name's designated resolver, and then query that resolver for 
%the record 
%associated with the name. Resolver contracts are allowed to 
%access the storage 
%of the ENS Registrar Controller, which means they don't have 
%to perform another 
%transaction for the resolver to know that a new record has 
%been created by the 
%ENS Registrar Controller. 
%While it is possible to enumerate 
%all of the transactions that recorded new names from the 
%Registrar Controller 
%Contract (and its historical predecessors, such as a 
%contract that was used to 
%register short names earlier in ENS's lifetime), this 
%approach still yields some
%nodes for which names were never recorded. 


Figure~\ref{fig:ens_records} shows the distribution of the types of records 
stored in ENS for our sample of names. The majority resolve 
to wallet addresses or text 
records, not IP addresses, traditional domains, or DS addresses. We broke down 
the text records, which are key/value pairs, by the most common key names: 
these keys are marked with an asterisk. Only the most common 25 keys are shown. 
We note that only 17 names had \texttt{dns\_wire\_format} records, which are 
intended to store traditional DNS 
records, and all 17 are malformed as far as we can tell. 
%\randall{should maybe 
%	work on that some more? Couldn't tell what was wrong by 
%examining the 
%	octets.}

%Describe the registrar/registry/resolver structure. We took a sample 
%of X domains from the most 
%frequently updated resolver contracts.
%
%Haven't found anything bad yet except loli-hentai.x. This 
%system, like all other uncensorable systems, will probably 
%eventually attract CSAM. Describe what we did to find 
%domains, how I crawled a subset.

\subsubsection{Unstoppable Domains}
\label{sec:unstoppable_overview}

\begin{figure}[t]
	\centering
	\includegraphics[width=3in]{figs/all_unstoppable_records.pdf}
	\caption{Records stored by Unstoppable Domains names.}
	\label{fig:unstoppable_records}
\end{figure}

%Describe the registry contract and the crawled domains (I did 
%crawl these didn't I?)

Like ENS, Unstoppable Domains uses Ethereum smart contracts as 
registrars. Unstoppable Domains names are divided into two systems. CNS 
(the Crypto Name System) contains all names with \texttt{.crypto} 
alt-TLDs, and has separate registry and resolver contracts. Later, Unstoppable 
added UNS 
(the Unstoppable Name System), which simplified name resolution by 
combining the resolver and registry contracts, and added several 
new alt-TLDs. Unstoppable Domains names never have to be renewed; they are 
purchased once and then owned indefinitely.

Like ENS names, Unstoppable Domains names are referenced by their hashes. We 
extracted all hashes from the UNS and CNS registry contracts 
by searching all of their transactions, and then found each 
hash's name and records by querying Unstoppable Domains' 
metadata endpoint.\footnote{	
\texttt{https://metadata.unstoppabledomains.com/metadata/}} 
This approach yielded a sample of 16,026 names. As with ENS, 
some names appear to exist in Unstoppable Domains that cannot 
be found by collecting transactions from these registry 
contracts. For example, it appears to be possible to store Unstoppable Domains 
names on the Polygon blockchain instead of Ethereum. We therefore note that our 
sample does not contain all of the names present in Unstoppable Domains.

Figure~\ref{fig:unstoppable_records} shows the distribution 
of record types found in the Unstoppable Domains names. As in 
ENS, the majority of names have wallet records rather than 
records that point to websites in any way. The second most common 
type of record is ``whois.for\_sale.value,'' showing that many 
names are seen as speculative assets. Unstoppable Domains also 
provides an easy way for users to link to IPFS records.

We performed a web crawl of all of the Unstoppable names that 
had records pointing to websites, whether IPFS records, 
traditional IP addresses, or traditional domains. We took 
screenshots of the 367 websites we arrived 
at, inspected them manually, and did not find any evidence of 
malware use. Most 
websites were personal sites, Web3-based business sites, or related to the sale 
or collection of NFTs. 



%Interestingly, we note that Unstoppable Domains has claimed that their naming 
%system cannot be easily abused, for two reasons. First, Unstoppable Domains 
%claims to police which domains may be sold. Unstoppable Domains claimed to 
%have 
%``prevented the registration of domains associated with known pirating 
%software 
%or other types of IP theft and fraud''~\cite{pegoraro_blockchain_2021}. 


\section{Accessing records}
\label{sec:accessing_records}

\begin{figure*}[t]
	\centering
	\includegraphics[width=\textwidth]{figs/intervention_locations.pdf}
	\caption{Potential locations of interventions for 
	blocking access to DNS-based and blockchain-based C2 
	server names.}
	\label{fig:malware_contacting_cnc}
\end{figure*}

We have established that malware requires a naming layer to access its C2 
centers, and 
that blockchain-based naming systems are attractive options for implementing 
such a naming layer. We have also examined several naming systems individually. 
We now argue that blockchain-based naming systems share certain common 
challenges, and the way these systems overcome these challenges presents 
different advantages or disadvantages for defenders. The first 
of these fundamental challenges is accessing the blockchain.

Accessing any distributed, peer-to-peer system for the first time requires 
learning the address of at least one participating node. In general, there 
are two methods for finding such an address: connecting to a 
proxy that already knows how to reach a member of the system, or acting 
as a full member of the system and utilizing its peer-to-peer discovery 
protocol. The latter approach requires knowing a list of ``bootstrap nodes.'' 
For example, Ethereum uses a list of bootstrap nodes that are hard-coded into 
client implementations~\cite{geth_bootstrap}. Bitcoin 
stores lists of bootstrap nodes in DNS TXT records maintained by volunteers, as 
well as hard-coded 
lists~\cite{bitcoin_bootstrap}. A third option, a 
local discovery protocol that floods the network with 
messages looking for nodes, has not been adopted by any 
blockchains that we are aware of. Such an approach would only be successful if 
nodes running the blockchain were present on most networks. 
 
Figure~\ref{fig:malware_contacting_cnc} compares the steps an infected host 
takes to resolve a DNS C2 domain to the steps required to resolve a blockchain 
name. It also details the interventions that can be taken by defenders at each 
step. We now describe each step in detail.

\subsection{Reaching the resolver}

Regardless of whether a request is destined for DNS or a blockchain naming 
system, it must first reach the machine that acts as a resolver: either the DNS 
resolver or the proxy in Figure~\ref{fig:malware_contacting_cnc}. Defenders may 
be able to intervene before this point by placing middleboxes with filter lists 
in the network. Some networks already have such defenses: for example, some ISP 
networks redirect all DNS requests to the ISP's own resolver, which can 
implement a filter list. This defense is probably not currently intended to 
block blockchain names, but it works by chance in some cases. Some malware uses 
ordinary DNS rather than DNS-over-HTTPS to request blockchain domains, under 
the assumption that the proxy the query is intended for will redirect it to the 
blockchain naming system in the correct format. When ISPs perform DNS 
redirection to their own resolvers, these queries get redirected to the DNS 
root, which cannot resolve the alt-TLDs used by blockchain naming systems. We 
present our study of this phenomenon in Section~\ref{sec:b-root}. We observe 
that filter lists are only a partial defense against malware, because malware 
may utilize DGAs to evade them. 

\subsection{Interventions at the name resolver}

When using blockchain naming systems, the entity that first attempts to resolve 
a request is a proxy rather than a pure DNS resolver. It may expect queries in 
the form of DNS-over-HTTPS, unencrypted DNS, or in an 
arbitrary format. Instead of querying the DNS root zone, the 
TLD resolver, and eventually the authoritative nameserver to 
resolve a name, the proxy must connect to the blockchain and 
retrieve the record from one of the participants.
Defenders may intervene at a traditional DNS resolver by requesting that the 
resolver implement a filter list, or the resolver operators may elect to 
implement one voluntarily. However, proxies that resolve blockchain names may 
be resistant to such voluntary efforts, because proxy 
operators often identify with the libertarian values 
associated with the goals of blockchain. 


Proxies are currently the most common method for resolving 
blockchain names. Table~\ref{tab:proxies_and_tlds} shows a 
selection of the proxies and tools that resolve names from 
each of the systems we study. The list of proxies is not 
exhaustive, but represents 
a subset of the best-known proxies in use at the time of 
writing. While most large 
browsers, such as Safari, Chrome, and Firefox, do not support 
any blockchain naming systems natively, some naming systems 
provide browser extensions that redirect blockchain name 
queries to proxies using DoH. A few browsers do resolve 
blockchain names without requiring extensions, such as Brave, 
which partners with a proxy called 
Infura~\cite{brave_uses_infura}. Some naming systems have 
partnerships with existing DNS resolvers. For example, 
NextDNS's DNS resolvers can act as proxies to resolve 
Handshake names. Finally, some naming systems, such as 
Handshake, also provide stub resolver implementations that 
run locally on a user's computer. These stub resolvers also 
work by routing blockchain name queries to proxies.

All of these proxies are centralized, in that they 
are controlled by a single authority. This is good 
news for defenders: similarly to traditional registrars, they 
are vulnerable to legal takedowns. They can be served with 
TROs or warrants and compelled to stop giving access to 
abused domains, as long as they operate within a jurisdiction 
amenable to such efforts. A centralized proxy could also be 
neutralized by serving a takedown order to its 
hosting provider, although this approach would produce 
varying amounts of the collateral damage depending on how 
many licit users utilize the proxy. 
While these interventions are not 
foolproof, they are subject to the same advantages and 
disadvantages as interventions on traditional registrars. 
Thus, centralized proxies return the distributed naming 
ecosystem to a state similar to the DNS ecosystem, from a 
defender's point of view. 
%We conclude that proxy takedowns 
%are a viable method for 
%defenders to intervene in the blockchain DNS ecosystem. 
However, while proxies greatly simplify the process of 
connecting to a blockchain, they are not strictly necessary, 
which is bad news for defenders.

\subsection{Skipping the proxy: the rise of light clients}

We initially assumed that no infected host would be able to 
skip the proxy and participate directly 
in the blockchain, because acting as a blockchain node 
requires too many resources. However, this assumption turned 
out to be incorrect, because of the rise of \emph{light 
clients}. When blockchains were first envisioned, most 
assumed that every participant in the network would be a 
``full'' implementation of a node: it would contain 
enough state to reconstruct the entire history of the chain, 
all the way back to the first transaction. Additionally, each 
node would contribute to the 
blockchain by verifying every transaction it heard about. As 
blockchains grow over time, they become too 
resource-intensive to run on anything other than a dedicated, 
powerful machine. Two resources serve as 
the constraints: first, CPU power, which is obviously 
necessary to perform mining but now is even a bottleneck for 
transaction verification, because so many transactions happen 
per second~\cite{citation_needed}. Second, disk 
space and speed. For example, a full Ethereum node cannot be 
run on a machine with a hard disk drive anymore, because 
nothing slower than an SSD can keep up with the 
reads and writes required~\cite{citation_needed}. These 
resource constraints make it 
very unlikely that malware could run ``full'' blockchain 
nodes on infected hosts. However, these constraints have also
given rise to the concept of a ``light client,'' a blockchain 
node with limited functionality that can fetch transactions 
from the chain but does not contribute by verifying 
transactions, mining, or broadcasting. Light clients are 
designed to run on laptops and mobile devices. As such, they 
use few enough resources to reasonably be included in 
malware. 


\subsection{Interventions at the Database Locator}

Light clients enable malware to act as a first-class member 
of a blockchain, and discover other members of the chain 
using the chain's peer-to-peer discovery protocol without 
using a centralized proxy. In this case, defenders are left 
with a harder location to stage an intervention: the 
blockchain's bootstrap nodes, which is the blockchain 
equivalent of a service that locates the database of naming 
records.

%Most peer-to-peer discovery 
%protocols allow a client to connect to the blockchain for 
%the 
%first time by hard-coding a set of ``bootstrap nodes.'' In 
%Ethereum, these bootstrap nodes are hard-coded into various 
%implementations of the clients, and in Bitcoin, they are 
%either hard-coded or accessible as TXT records stored at 
%various trusted domains. The list of bootstrap nodes may 
%also 
%be configured by the user. If the 
%malware chooses to use the bootstrap nodes that are 
%hard-coded by default into the light client implementation, 
%this may present a challenge for defenders, because taking 
%down those bootstrap nodes may cause collateral damage to 
%legitimate users attempting to join the chain. As such, 
%bootstrap nodes are a more difficult place to stage an 
%effective defender intervention than centralized proxies. 


In traditional DNS, the resolver must locate the database 
that contains a record by first querying the hierarchy of DNS 
servers: first the root and then the TLD resolver. The TLD 
resolver's role is to tell the DNS resolver which machine 
stores the database that ultimately contains a name's 
records. In a blockchain system, this role is filled by the 
blockchain's bootstrap nodes, which are publicly listed 
nodes that maintain connections to some of the other 
participants in the blockchain. The purpose of the bootstrap 
nodes is to provide a gateway to the blockchain for new 
participants: new blockchain nodes find their initial list of 
potential peers by connecting to the bootstrap nodes.

When defenders perform interventions by putting legal 
pressure on registrars, the intervention takes effect at the 
TLD resolver, which implements the changes to the zone file 
that affect the malware's domains. These changes can include 
``sinkholing'' the domain by causing it resolve to an IP 
controlled by defenders or ``freezing'' it so that 
its records cannot be modified. This intervention does not 
translate well to blockchain naming systems for several 
reasons. 

First, while bootstrap nodes are responsible for finding the 
entire naming database, they do not allow defenders to 
specify which blockchain systems a client may access and 
which it may not. This means that seizing a specific naming 
record, or even the entire naming system, is not possible at 
the bootstrap nodes. Consequently, disabling or seizing 
bootstrap nodes prevents all new clients from accessing any 
functionality provided by the blockchain, including the 
blockchain's cryptocurrencies and any services it offers 
unrelated to naming. This approach therefore carries the 
potential for a lot of collateral damage. Second, bootstrap 
nodes may be widely distributed across the globe, leading to 
jurisdictional challenges in bringing legal pressure to bear 
on their operators. Bootstrap nodes may also be difficult to 
find, since they may not be run by hosting providers but 
rather by anonymous individual volunteers. Third, bootstrap 
nodes may be numerous enough that finding and seizing them 
all may be prohibitively difficult. Finally, while the 
default bootstrap node lists are published for each 
blockchain, users may choose to substitute their own. A 
malware author could design a payload that contains an 
extensive list of machines that participate in a blockchain 
naming system, which would complicate a defender's efforts to 
take down all the potential participants. 

However, defenders could fall back to a blocklist-based 
approach to deny access to bootstrap nodes. For example, 
IDSes, enterprise firewalls, or ISP routers can drop traffic 
intended for bootstrap nodes. This approach is very similar 
to blocking any other 
malicious IP addresses, and is subject to the usual 
challenges. Defenders must keep blocklists up-to-date as 
malware authors update the IPs they connect to. To the 
advantage of defenders, any time malware authors 
are forced to update the IP addresses that bootstrap nodes 
may be found at, they run afoul of the ``sunk cost'' problem 
where infected machines that cannot be updated become
useless. A similar argument applies if malware chooses to 
access bootstrap nodes using hard-coded DNS domain names 
instead of hard-coded IP addresses. Additionally, 
traditional interventions against domain names apply in that 
situation as well. Thus, while intervening at bootstrap nodes 
poses more of a challenge than intervening at centralized 
proxies, defenders still have viable options to choose from.

\subsection{Interventions at the Database}

In traditional DNS, defenders can sinkhole the domain of an 
authoritative nameserver or seize the server itself to 
prevent malware accessing a C2 domain record. This 
intervention is impractical for blockchain names, 
because instead of a single machine acting as the 
authoritative nameserver, every blockchain node has a copy of 
the database. Seizing the database would require taking down 
every machine in the blockchain, which is an attack that 
blockchains are highly robust against.

\subsection{Interventions after the name record is acquired}

If an infected host successfully retrieves its C2 record, 
that record might take several forms. The three that we 
observed in existing blockchain naming systems that might be 
useful to malware were IP addresses, traditional DNS domains, 
and addresses for distributed storage systems like IPFS and 
SkyNet. We refer to these addresses as ``distributed 
storage'' (DS) addresses from this point forward. Some naming 
systems also allow users to store arbitrary text as 
records, which would let malware operators store nonstandard 
record types like links to social media posts. 

Each of these record types are subject to all of the 
traditional interventions that have already 
been described, except one: DS addresses. Distributed storage 
systems provide a limited form of ``bulletproof'' hosting, 
because they do not offer the ability to host dynamic 
websites. Any C2 center implemented entirely on such a 
system must be a simple file with no 
dynamic content. Additionally, today's DS systems are 
content-addressed, which requires malware operators to use a 
naming layer rather than hard-coding the DS address of 
their C2 server into their payloads. If the file that 
represents the C2 server changes, its address will change as 
well, because the address is created from a hash of the file.

In another advantage for defenders, some DS systems do not 
have redundancy: only a single machine hosts each piece of a 
file. This raises the possibility of 
discovering the particular 
machine responsible for hosting a C2 server and seizing 
it. 

%Accessing a DS system is the same as accessing a distributed 
%blockchain-based naming layer: you can do it either by proxy 
%or by being a first-class member of the network, in which 
%case you need bootstrap nodes. Theoretically, malware can go 
%straight to the storage layer, but this is unlikely to be 
%feasible given the way current DS systems are set up. The 
%reason malware needs 
%to use a naming layer to connect to the current generation 
%of 
%distributed file storage systems is because they're 
%content-addressable. As soon as the file changes, so 
%does its address, and the malware that knew the old address 
%is useless. 
%However, this is not a fundamental limitation of distributed 
%storage systems --- a system could be designed in the future 
%that doesn't have this limitation.

%\subsection{Old stuff, Accessing blockchains using proxies}

%As a side note, the existence of competing naming systems in 
%which name 
%collisions are possible 
%opens up the possibility of collision-based attacks. There 
%is no governance 
%system in place to prevent different systems from adopting 
%the same TLDs and 
%causing name collisions. In fact, several collisions already 
%exist, both 
%between blockchain-based systems themselves (Emercoin and 
%Unstoppable both 
%provide a ``.coin'' TLD) and between ICANN TLDs and 
%blockchain-based systems 
%(both Handshake and ICANN provide ``.music''). 
%This means the record that a user receives when resolving a 
%name with a 
%collision depends on which resolution method they use. If 
%the user has 
%multiple 
%proxies set up --- for example, multiple browser extensions 
%installed that 
%each 
%try to resolve domains with a specific TLD --- then the 
%record received will 
%depend on which of those proxies takes precedence, which is 
%not at all 
%clearly 
%defined. This lack of centralized governance enables several 
%classes of 
%attacks, such as squatting 
%and phishing, which are beyond the scope of this paper.



\subsection{Case studies}

\subsubsection{OpenNIC ceasing support of \texttt{.bit}}

OpenNIC is a decentralized proxy service that 
resolves names from several alternative naming systems, 
including Namecoin and Emercoin~\cite{opennic}. OpenNIC's 
resolvers are run by a small community of 
volunteers~\cite{opennic_servers}. In June 2019, this 
community voted of their own volition to remove support for 
Namecoin's \texttt{.bit} alt-TLD, because providers were 
beginning to block OpenNIC resolvers that were used by 
malware to resolve Namecoin 
names~\cite{opennic_namecoin_vote}.

OpenNIC's decision to cease supporting Namecoin was not the 
result of a direct intervention by defenders, but it still 
yields an important lesson. Even a decentralized proxy 
service may be composed of few enough individuals that it is 
possible to cajole or compel them to stop resolving names 
used by malware. Furthermore, OpenNIC's community held this 
vote in response to pressure from Spamhaus, Malwarebytes, and 
other providers, who began blocklisting OpenNIC resolver 
domains: even if a proxy's community cannot be contacted 
directly, it is still possible to pressure them to cease 
resolving names used by malware.

\subsubsection{BDNS takedown}

Defenders have already attempted to take down a proxy for 
accessing blockchain domains: ``Blockchain-DNS.info,'' also 
known as BDNS. BDNS reported on their website that in April 
2018, seven of their domain names were ``un-delegated'' and 
one of their API servers was shut down without 
warning~\cite{blockchain-dns-info-wayback}. The service 
received a message from Spamhaus shortly after noticing the 
takedown, stating that several of 
BDNS's domains had been added to Spamhaus's blocklist. BDNS 
claimed that their browser extensions continued to resolve 
blockchain names using other endpoints, and directed users to 
a list of endpoints that were still 
working~\cite{github_bdns_wayback}. BDNS also stated that 
they had moved some infrastructure to a friendlier hosting 
provider, PRQ, which states on its website that ``If 
[content] is legal in Sweden, we will host it, and will keep 
it up regardless of any pressure to take it 
down''~\cite{prq}. 

This takedown effort provides several lessons for defenders. 
First, defenders must take care when choosing a takedown 
strategy for a proxy. In this case, defenders tried two 
tactics: 
adding the proxy's endpoints to a widely used blocklist and 
taking down some domains and a hosting server entirely. 
The former tactic appeared to work well in locations where 
ISPs use Spamhaus's blocklist: BDNS stated that their 
proxy ``may be still unreachable in those parts of the 
world.'' However, the domain takedown appeared to be only 
partially effective, since BDNS could still resolve 
blockchain names using unaffected endpoints. We conclude that 
care must be taken to enumerate all of a proxy's endpoints 
and shut them down simultaneously.  

At the time of writing, all of the endpoints listed in BDNS's 
Github repository~\cite{github_bdns_wayback} are either 
failing to resolve or resolving but failing to 
load content. However, it is unclear whether defenders 
performed a more successful takedown or the service shut down 
for other reasons. One endpoint, \texttt{bdns.io}, has 
apparently been sinkholed by ShadowServer, since its NS 
records now point to variants of the name 
sinkhole.shadowserver.org.\footnote{sinkhole-0[0-4].shadowserver.org
and sinkhole-[a-b].shadowserver.org.} We were unable to 
determine if ShadowServer was involved in the original 
takedown effort or if it sinkholed this domain at a later 
date.


%\subsection{Naming record formats}
%
%
%
%
%\randall{where do I put this? Additionally, because all 
%blockchain 
%	records 
%	are public, anyone can fetch those records including 
%	defenders. You could theoretically scrape a blockchain 
%looking for records 
%	that match a known 
%	malicious format or with malicious traits of some sort 
%(owned by the same 
%	wallet?) and try to seize 
%	whatever those records point to.} 
%
%
%
%I don't know if IPFS/SkyNet have light nodes that could be 
%part of a malware payload, but there does seem to be a trend 
%in that direction as chains get heavier.
\input{case_studies}
\section{Measurements of Name Resolution Queries}
\label{sec:b-root}
% this is the b-root section

Because blockchain names require alternate resolution systems, we predicted 
that many of them would ``leak'' into the DNS when misconfigured machines 
attempt to resolve them as ordinary DNS domains. An observer might therefore be 
able to see which names are popular by measuring which blockchain names are 
requested through ordinary DNS requests. We took two 24-hour samples of names 
with Namecoin and Emercoin's alternate TLDs that appeared in 
queries to a root DNS server. One of these samples was from October 19, 2021, 
and the other was from April 15, 2022.

Discovering which names users are resolving in blockchain-based 
naming systems is challenging: methods that work in traditional DNS, such as 
cache snooping, rely on knowing which domains to look for, and reads within a 
blockchain system don't leave traces in cache the way lookups to ordinary 
resolvers do. We hypothesized that some attempts to resolve blockchain names 
would be misconfigured and might leak into regular DNS. To test this 
hypothesis, we obtained samples of the queries for domains with BNS alt-TLDs 
from one of the root DNS resolvers, B-root.

\subsection{Removing irrelevant queries}

Some queries for domains with alt-TLDs at B-root were unrelated to the 
blockchain systems they share alt-TLDs with. A prior study on root DNS queries 
found that some networks use non-ICANN TLDs internally under the assumption 
that queries for those names will never reach external DNS resolvers. However, 
these queries frequently leak to external networks~\cite{chen_wpad_2016}. We 
observed that some 
internal networks use alt-TLDs that overlap with the blockchain naming systems' 
alt-TLDs. For example, we observed numerous queries for subdomains of 
\texttt{kroger.com.btc}, \texttt{nordic.x}, \texttt{sata.x}, and others. These 
domains are not registered in the Unstoppable Domains namespace. We removed 
names from the B-root sample if they were not registered in their respective 
blockchain-based naming system.

We also discovered that domains used in the WPAD discovery protocol, a protocol 
used for local network participants to discover DNS proxies, used alt-TLDs that 
overlapped with blockchain DNS. WPAD DNS lookups take the form of 
\texttt{wpad.company-name.alt-TLD}~\cite{chen_wpad_2016}. If we observe a 
request of that form, we consider it to be strong evidence that other requests 
to subdomains of \texttt{company-name.alt-TLD} have escaped from an internal 
network. We remove these requests from our sample as well.

After removing names that are not registered in Unstoppable Domains, we 
observed no more than six names with each Unstoppable TLD requested from B-root 
on either of our sample days. Either no one is using Unstoppable to look up 
domains (consistent with the types of records they store), or Unstoppable has 
modest adoption but most people are using the extensions properly.

ENS names leaked slightly more frequently: we saw 186 names in the October 19th
sample and 313 in the April 15th sample. 

But what about the unregistered domains?

\subsection{Queries for Namecoin and Emercoin Names}

\begin{table}
	\begin{tabular}{lrr}
		\toprule
		Malware & Domain & Lookups \\
		\midrule
		Gandcrab	&	malwarehunterteam.bit	&	696	\\
		&	politiaromana.bit	&	682	\\
		&	gdcb.bit	&	632	\\
		&	zonealarm.bit	&	1256	\\
		&	ransomware.bit	&	2078	\\
		&	nomoreransom.coin	&	2242	\\
		CHESSYLITE	&	leomoon.bit	&	1870	\\
		&	lookstat.bit	&	1420	\\
		&	sysmonitor.bit	&	1038	\\
		&	volstat.bit	&	910	\\
		&	xoonday.bit	&	1146	\\
		Dofoil	&	vrubl.bit	&	1976	\\
		&	levashov.bit	&	2118	\\
		&	vinik.bit	&	12530	\\
		KPOT Stealer	&	kpotuvorot10.bit	&	3902	\\
		&	star-fox.bit	&	701	\\
		Team9 Loader	&	bestgame.bazar	&	1884	\\
		&	forgame.bazar	&	1730	\\
		&	zirabuo.bazar	&	102	\\
		&	tallcareful.bazar	&	292	\\
		&	coastdeny.bazar	&	278	\\
		BazarLoader	&	acegikbcggin.bazar	&	1092	\\
		&	acegilbcggio.bazar	&	934	\\
		Trojan RTM	&	stat-counter-[0-9]-[0-9].bit	&	20996	\\
		Necurs	&	jfbbrj3bbbd.bit	&	3009	\\
		\bottomrule
	\end{tabular}
	\caption{Examples of malicious Namecoin and Emercoin domains in the October 
	sample of B-root 
		queries.}
	\label{tab:namecoin_emercoin}
\end{table}

\begin{figure}[t]
	\centering
	\includegraphics[width=3in]{figs/namecoin_and_emercoin_tlds_num_names.pdf}
	\caption{Number of unique Emercoin and Namecoin names requested in day-long 
		samples from B-root.}
	\label{fig:namecoin_and_emercoin_names}
\end{figure}

\begin{figure}[t]
	\centering
	\includegraphics[width=3in]{figs/namecoin_and_emercoin_tlds_num_counts.pdf}
	\caption{Total requests for Emercoin and Namecoin names in day-long samples 
		from B-root.}
	\label{fig:namecoin_and_emercoin_counts}
\end{figure}

 We examined the names from each sample 
and found ample evidence of names that are likely to be used by malware. We 
manually inspected 
the Emercoin and Namecoin names with the most queries, and found that the vast 
majority of the most 
frequently queried names were associated with malware. An example of malware 
names from the October 
sample of B-root queries is shown in Table~\ref{tab:namecoin_emercoin}. We 
found that many of the 
names appeared to be randomly generated, making them unlikely to be benign. 
\randall{not true, randomly generated could be non-malware we just don't know 
	what they are.} The only popular Emercoin 
or Namecoin name that was not associated with malware was 
\texttt{nnm-club.lib}, which appears to be a 
pirating site. These findings are consistent with previous work that found the 
vast majority of names 
and IPs on Emercoin and Namecoin to be associated with malicious 
behavior~\cite{patsakis_unravelling_2020, casino_unearthing_2021}. 
\randall{more results from these 
	papers?}

\subsection{Queries for UD and ENS names}

%Second, we note that some of these names appear to be intended for typo 
%squatting. \randall{Tens of? Hundreds of?} lookups occur for variations of 
%popular names such as ``netflix,'' ``gmail,'' ``apple,'' and 
%``yahoo.'' The string ``netflix'' in particular was popular. \randall{All of 
%these were .kred and 
	%.luxe and stuff, not Namecoin/Emercoin.
	
\subsection{.x}

When you remove the unregistered Unstoppable and ENS domains and the wpad 
domains, there’s like ONE .x domain left. I guess a lot of the .x domains were 
either misconfigured IP addresses (2218, found with 
commented regex) or randomly generated names (6956, according to my heuristic 
of no English words above either 4 or 5 chars (check this, probably redo this 
number)) %([0-9]+|x+)\.([0-9]+|x+)\.([0-9]+|x+)\.x\.\s

\subsection{.bit and .bazar}

.bit names have a huge number of lookups for a small number of names. This is 
in contrast to the randomly generated .eth and .x names, the vast majority of 
which have two lookups each. Since we are pretty sure the majority of .bit 
names are malware-related, and having tons of lookups is consistent with how 
DGAs work, we predict that the randomly generated .x, .eth, and .bitcoin 
domains are not attempts to find C2 domains.



%\begin{figure}[t]
%	\centering
%	
%\includegraphics[width=3in]{figs/ud_and_ens_tlds_num_names_no_wpad_or_unregistered_ens_ud.pdf}
%	\caption{Number of unique Emercoin and Namecoin names requested in day-long 
	%		samples from B-root.}
%	\label{fig:ud_and_ens_names}
%\end{figure}
%\begin{figure}[t]
%	\centering
%	
%\includegraphics[width=3in]{figs/ud_and_ens_tlds_num_counts_no_wpad_or_unregistered_ens_ud.pdf}
%	\caption{Total requests for Emercoin and Namecoin names in day-long samples 
	%		from B-root.}
%	\label{fig:ud_and_ens_counts}
%\end{figure}
\section{Discussion}
\label{sec:discussion}

Out of the five naming systems we examine, we find none so far that present an 
entirely intractable problem for defenders. For a naming system to present a  
threat, it must be both easily usable by malware authors and 
popular enough that blocking its bootstrap nodes, or blocking 
access to it entirely, will cause significant collateral damage to licit users.
For a system to be widely adopted by licit users, it must have three necessary 
characteristics. 

First, the system's name management must be as 
easy or easier than name management on traditional DNS domains. Users must not 
be required to write 
code themselves to interact with smart contracts, as is currently the case with 
each of the systems we 
study if the user does not use a custodial wallet. Users also 
must not be required to run a blockchain node in order to 
manage their names, as 
Handshake currently requires to the best of our knowledge.

Second, the transactions that are required to register and update names must be 
affordable. 
Transactions on Ethereum, in our experience, cost anywhere between \$60 and 
\$140 during the course of 
our experiments, although we discovered that we were attempting to make 
transactions during periods of 
high network congestion and fees were unusually high. Even transaction fees as 
low as ten dollars per 
transaction are far less affordable than transaction fees on naming-specific 
chains, which can be as 
low as a few cents. This dynamic may make ordinary users more likely to embrace 
naming systems built 
on naming-specific chains, rather than general-purpose chains. However, 
general-purpose chains may be 
better known, and therefore more likely to be trusted by users even if 
transaction fees are higher 
than on naming-specific chains. A trade-off may therefore exist between 
affordability and perceived 
trustworthiness and name recognition.  

Third, licit users are unlikely to embrace any naming system that does not have 
widespread browser 
adoption. Browser adoption is hindered by naming systems' lack of coordination, 
which currently leads 
to name collisions: for example, the alt-TLDs 
\texttt{.wallet}, 
\texttt{.coin}, and \texttt{.x} 
are currently used by multiple blockchain naming systems. Some newly created 
ICANN TLDs also collide 
with Handshake TLDs, such as \texttt{.music}. Naming collisions present a 
barrier to browser adoption 
because the browser would either have to enforce some sort of precedence for 
systems that include 
colliding names, or users would have to choose which naming system to use for 
each name with 
collisions. Either option will confuse and frustrate users who are unfamiliar 
with the concepts of 
namespaces. So far, only browsers that focus on privacy as one of their primary 
features have chosen 
to resolve alternate naming systems, and none have chosen to resolve systems 
that might collide with 
either each other or ICANN TLDs. Until browsers can resolve an alternate naming 
system natively, users 
are unlikely to adopt that naming system.

We conclude that the higher ease of use of purchasing, 
modifying, and resolving traditional DNS 
domains is a very high barrier for blockchain-based naming 
systems to overcome. As long as blockchain naming systems are 
not widely adopted, we predict they will not become entirely 
intractable problems for defenders.
\section{Related Work}
\label{sec:related}

% Kalodner 2015, An Empirical Study of Namecoin
Kalodner et al. performed the first study to our knowledge of 
Namecoin in 2015~\cite{kalodner_namecoin_2015}. They conclude 
that the Namecoin ecosystem was ``dysfunctional:'' only 28 
out of 120,000 registered names were valid, not squatted, and 
had nontrivial content.

% Unravelling Ariadne's Thread
Patsakis et al. present an analysis of potential weaknesses 
and user risks of 
Namecoin and Emercoin, including the risks of squatting, 51\% 
attacks, phishing, and abuse by 
malware~\cite{patsakis_unravelling_2020}. The authors also 
provide an overview of the names stored in these systems, and 
found that many names registered in the Alexa Top 1K were 
also registered under Namecoin and Emercoin's alt-TLDs. Most 
of these squatted names redirected to pornographic websites.

% Unearthing malicious campaigns and actors from the 
% blockchain DNS ecosystem~\cite{casino_unearthing_2021}, 
% good graphs.
Casino et al. analyzed the IP addresses in Namecoin and 
Emercoin records~\cite{casino_unearthing_2021}. They first 
identified malicious IP addresses using several threat 
intelligence databases, and then clustered all the IPs into 
``malicious,'' ``suspicious'' and ``benign'' categories with 
a ``poisoning'' 
approach. An IP was labeled ``malicious'' if a threat 
intelligence database categorized it as such. It was labeled 
``suspicious'' if it appeared in the same wallet, was 
resolved to by the same domain, or shared the same email TXT 
record as a malicious IP, and ``benign'' if it had no 
connection to a malicious IP. Casino et al. discovered that 
only 8\% of the IPs in Emercoin and 28\% of those in Namecoin 
had no association with malicious IP addresses. While this 
paper mentioned the existence of more recent blockchain 
naming systems, it did not perform an analysis of any system 
except Namecoin and Emercoin.

%Do we need papers that have tried to make decentralized 
%naming systems? 
%Ariadne's Thread has a list.

Numerous other blockchain-based naming systems have been 
proposed, including
the Blockstack Naming System~\cite{ali2016blockstack}, 
Bitforest/Conifer~\cite{dong2018bitforest, dong2018conifer}, 
BlockDNS~\cite{blockdns}, and 
Nebulis~\cite{nebulis_2016}. To our knowledge, 
only Blockstack has evolved into a commercial product. We 
excluded the Blockstack Naming System from this work because 
it does not appear to be as popular as the other systems we 
study.

Other work has analyzed the ways in which blockchain 
technologies in general might be abused by 
malware. Pletinckx et al. analyzed Cerber ransomware and 
found that it used blockchain wallet addresses as 
domains~\cite{plentinckx_cerber_2018}. 
Hassan et al. point out that blockchain nodes reside 
in so many different legal jurisdictions, it will be 
difficult for regulators to control what information gets 
passed across country borders~\cite{hassan_blockchain_2020}. 
Moubarak et al. present a theoretical design for malware to 
store pieces of its payload on 
Bitcoin~\cite{moubarak_developing_2018}.
%Dai et al. enumerated several methods by which 
%blockchains can be used as attack 
%vectors~\cite{dai_cybersecurity_2017} \randall{wrong paper}

Relatively little work has been done on defenses against 
malware that uses blockchain naming systems. Huang et al. 
developed a machine learning-based detection 
method for distinguishing malicious blockchain-based names 
from benign names in DNS traffic~\cite{huang2020leopard}.
Hu et al. presented a brief comparison of DNS and 
Bitcoin-based naming systems, and noted that small, 
naming-specific blockchains like Namecoin were vulnerable to 
51\% attacks~\cite{wei2017review}.

Prior work has evaluated the effectiveness of interventions 
that target DNS domains. Kesari et al. provide an overview of 
legal intervention methods and cites their use in a number of 
malware takedowns~\cite{kesari_deterring_2017}. Wang et al. 
studied the use of TROs to seize storefronts run by 
spammers~\cite{wang_search_2014}. Liu et al. analyzed the 
effectiveness of two interventions that were initiated by 
registrars and designed to stop 
spammers from registering 
domains~\cite{liu_registrar_intervention}. Prior literature 
has also analyzed interventions based on taking down hosting 
providers, and concluded that these interventions have modest 
or mixed effectiveness~\cite{bradbury2014testing, 
konte2015aswatch, noroozian_platforms_2019, alrwais_bph}. 
\section{Conclusion}

While decentralized naming and hosting systems pose challenges, they cannot 
entirely 
eliminate their reliance on systems with centralized authority. Whenever 
malware uses a centralized 
resource to enable its use of decentralized ones, defenders can intervene. 
Defenders cannot serve 
legal takedown orders to a centralized registrar to take 
down a blockchain domain, but they can prevent malware from accessing the 
blockchain in the first 
place, or target the DNS domain or IP address that the blockchain domain 
resolves to. We examined existing blockchain-based naming systems and found 
that naming systems on general purpose blockchains doare not 
currenly attractive to malware because of their high cost. In 
contrast, systems on naming-specific blockchains present an 
ongoing threat, but these systems have 
been well studied. They are also susceptible 
to defenses such as 
blocklisting every IP address stored in the name records, 
blocking the proxies that resolve the names, or blocking 
the system entirely, because so little licit 
content exists on those blockchains. We conclude that for a naming system to be 
truly more dangerous than DNS, it must achieve widespread 
adoption as well as inexpensive transactions and high 
ease-of-use, and no existing naming systems have yet achieved 
all three characteristics. 

%\section{Acknowledgments}
%
%The authors would like to thank Wes Hardaker, in whatever 
%format he wants the 
%b-root project cited, for the b-root names.

\bibliographystyle{IEEEtran}
\bibliography{references}

%\appendix
%\section{Ethics}
%
%This work raises no ethical concerns.

\end{document}
