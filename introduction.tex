\section{Introduction}

%\begin{itemize}
%	\item Malware, particularly botnets (any other types? ransomware?) 
%	uses DNS to reach CNC 
%	servers. 
%	\begin{itemize}
	%		\item Malware needs a naming layer because of the 
	%		sunk infrastructure cost: 
	%		any malware already deployed that uses an IP that gets 
	%blocked/taken 
	%		over is now useless. Malware authors want a record they can change.
	%		\item Naming layer needs to be hard to block at both 
	%		the request level and the system level, so that 
	%		already-distributed malware doesn't lose access and 
	%		become useless. 
	%	\end{itemize}
%	\item DNS is decentralized in that there are many resolvers, but 
%	centralized in that there are centralized authorities. Defenders can serve 
%	legal takedown notices to those centralized authorities to block malware's 
%	access to its CNC servers.
%	\item Pivot: Malware is starting to use a truly decentralized naming 
%	system, ``blockchain DNS.'' This creates several challenges for defenders:
%	\begin{itemize}
	%		\item No central authority is capable of enacting domain-level 
	%takedown 
	%		orders
	%		\item For large chains, there is enough legitimate content that 
	%		blocking access to the whole chain is infeasible
	%		\item Transactions on large chains cost a lot of money. This limits 
	%		defenders' abilities to stage interventions like pre-registering 
	%all 
	%		domains listed in a malware's DGA.
	%	\end{itemize}
%	\item We study the ecosystem and point out several places where 
%	interventions are still possible, and discuss the challenges to defenders 
%	and occasional advantages they gain when malware uses decentralized naming 
%	such as blockchain DNS.
%\end{itemize}

Malware that is distributed across multiple machines needs a way to distribute 
commands, upload stolen 
data, and coordinate between infected hosts. Most malware, such as botnets or 
ransomware, uses a 
central command and control (C2) server for this task. However, as a single 
point of failure, a 
central C2 server presents an obvious weak link for defenders to 
target~\cite{kesari_deterring_2017}. 
Malware authors must therefore be able to easily relocate and replace a C2 
server after a defender 
takedown. All previously infected hosts must be able to find the new server at 
its new address, 
without outside coordination --- if they cannot, they become useless. Malware 
authors avoid 
this ``sunk cost'' problem by providing a layer of indirection --- a naming 
layer --- instead of 
hard-coding a fixed C2 server address directly into deployed malware. This 
naming layer must be 
resilient to takedown efforts.

Until recently, the naming layer used most frequently by malware was ordinary 
DNS, which is rarely 
blocked at the protocol level, universally supported, and easy to configure. 
Malware authors use 
various strategies, such as DGAs (domain generation 
algorithms), 
to cycle through domains and complicate defense efforts. 
However, 
DNS domains are subject 
to central authorities (registrars), who may be compelled to 
seize or deny access to abused 
domains. Malware 
authors have recently come up with an innovative solution to this risk: they 
have started to use 
\emph{decentralized naming systems}, in particular naming systems built on 
blockchains. 

Blockchain naming systems present several potential 
challenges for 
defenders. First, because they 
have no central authority to carry out legal takedown 
requests, they are immune to one of the most effective tools 
in malware defenders' arsenals. Second, some blockchain 
naming systems 
have high transaction costs to register and manage domains, 
which renders some existing defense 
strategies ineffective. For example, registering all domains 
that a DGA can generate 
is impractical in expensive blockchain naming systems. 
However, blockchain naming systems 
present challenges to malware authors as well, such as how to stealthily access 
the system.

We study the blockchain naming ecosystem and point out 
several places where defender interventions are still 
possible. In particular, we argue that infected hosts must 
pass through infrastructure that is at least 
partially centralized to access any blockchain naming system, 
and these centralized or partially centralized 
``chokepoints'' still present viable locations to stage 
interventions. We also study five existing decentralized, 
blockchain-based naming systems, and discuss the challenges 
and advantages that each present to malware authors and 
defenders. We conclude that which blockchain 
naming systems present a significant threat, defenders still 
have viable options for enacting C2 takedowns.

The remainder of this paper is organized as follows. 
Section~\ref{sec:background} describes how malware currently 
uses DNS domains to contact C2 servers, how defenders 
stage interventions against these domains, and how blockchain 
naming systems render these interventions ineffective. 
Section~\ref{sec:naming_systems} gives an overview of the 
five most widely adopted blockchain naming systems. 
Sections~\ref{sec:accessing_records} describes where defenders 
can intervene in malware's attempts to access naming systems 
and control naming records. Section~\ref{sec:case_studies} 
lists three case studies of interventions against blockchain names and their 
lessons for defenders. Section~\ref{sec:b-root} details 
measurements of how blockchain naming systems are used by 
malware today. Section~\ref{sec:discussion} discusses the 
theoretical requirements for a naming system to be a threat, 
Section~\ref{sec:related} presents related work, and 
Section~\ref{sec:conclusion} concludes.