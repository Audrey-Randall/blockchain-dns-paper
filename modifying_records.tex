\section{Other challenges of blockchain naming systems for defenders and 
malware operators}
\label{sec:modifying_records}

\subsection{The price of names}

%The price is a factor when we talk about buying names and modifying them.
%
%In addition to ensuring that infected hosts can access a 
%blockchain naming system, malware operators must be able to 
%modify the naming records the system stores. If defenders 
%take down or block access to a particular C2 server's 
%address, malware operators must be able to quickly change the 
%C2 server's naming record. Any delay during which infected 
%hosts cannot connect to the C2 server may mean loss of 
%revenue for the malware campaign. However, modifying records 
%on blockchain naming systems is not necessarily as easy as 
%modifying DNS records. These challenges differ depending on 
%whether the underlying blockchain is naming-specific or 
%general-purpose.


%\randall{Does it take longer to propagate modifications 
%through a blockchain 
%	than through DNS? Is this a challenge for malware bc it 
%introduces 
%	downtime? 
%	Someone said it could take half an hour for a txn to be 
%confirmed on a 
%	blockchain.}



\randall{Out of place. Move to the relevant sections on accessing names?}


Additionally, low demand 
for these services from licit users enables interventions that would cause 
significant collateral damage on more popular blockchains. These interventions 
include blocklisting every record within the chain, blocking access to the 
chain entirely (e.g., from enterprise networks or ISPs' networks), or disabling 
its bootstrap nodes to prevent new peers from joining the blockchain.