\section{Related Work}
\label{sec:related}

% Kalodner 2015, An Empirical Study of Namecoin
Kalodner et al. performed the first study to our knowledge of 
Namecoin in 2015~\cite{kalodner_namecoin_2015}. They conclude 
that the Namecoin ecosystem was ``dysfunctional:'' only 28 
out of 120,000 registered names were valid, not squatted, and 
had nontrivial content.

% Unravelling Ariadne's Thread
Patsakis et al. present an analysis of potential weaknesses 
and user risks of 
Namecoin and Emercoin, including the risks of squatting, 51\% 
attacks, phishing, and abuse by 
malware~\cite{patsakis_unravelling_2020}. The authors also 
provide an overview of the names stored in these systems, and 
found that many names registered in the Alexa Top 1K were 
also registered under Namecoin and Emercoin's alt-TLDs. Most 
of these squatted names redirected to pornographic websites.

% Unearthing malicious campaigns and actors from the 
% blockchain DNS ecosystem~\cite{casino_unearthing_2021}, 
% good graphs.
Casino et al. analyzed the IP addresses in Namecoin and 
Emercoin records~\cite{casino_unearthing_2021}. They first 
identified malicious IP addresses using several threat 
intelligence databases, and then clustered all the IPs into 
``malicious,'' ``suspicious'' and ``benign'' categories with 
a ``poisoning'' 
approach. An IP was labeled ``malicious'' if a threat 
intelligence database categorized it as such. It was labeled 
``suspicious'' if it appeared in the same wallet, was 
resolved to by the same domain, or shared the same email TXT 
record as a malicious IP, and ``benign'' if it had no 
connection to a malicious IP. Casino et al. discovered that 
only 8\% of the IPs in Emercoin and 28\% of those in Namecoin 
had no association with malicious IP addresses. While this 
paper mentioned the existence of more recent blockchain 
naming systems, it did not perform an analysis of any system 
except Namecoin and Emercoin.

%Do we need papers that have tried to make decentralized 
%naming systems? 
%Ariadne's Thread has a list.

Numerous other blockchain-based naming systems have been 
proposed, including
the Blockstack Naming System~\cite{ali2016blockstack}, 
Bitforest/Conifer~\cite{dong2018bitforest, dong2018conifer}, 
BlockDNS~\cite{blockdns}, and 
Nebulis~\cite{nebulis_2016}. To our knowledge, 
only Blockstack has evolved into a commercial product. We 
excluded the Blockstack Naming System from this work because 
it does not appear to be as popular as the other systems we 
study.

Other work has analyzed the ways in which blockchain 
technologies in general might be abused by 
malware. Pletinckx et al. analyzed Cerber ransomware and 
found that it used blockchain wallet addresses as 
domains~\cite{plentinckx_cerber_2018}. 
Hassan et al. point out that blockchain nodes reside 
in so many different legal jurisdictions, it will be 
difficult for regulators to control what information gets 
passed across country borders~\cite{hassan_blockchain_2020}. 
Moubarak et al. present a theoretical design for malware to 
store pieces of its payload on 
Bitcoin~\cite{moubarak_developing_2018}.
%Dai et al. enumerated several methods by which 
%blockchains can be used as attack 
%vectors~\cite{dai_cybersecurity_2017} \randall{wrong paper}

Relatively little work has been done on defenses against 
malware that uses blockchain naming systems. Huang et al. 
developed a machine learning-based detection 
method for distinguishing malicious blockchain-based names 
from benign names in DNS traffic~\cite{huang2020leopard}.
Hu et al. presented a brief comparison of DNS and 
Bitcoin-based naming systems, and noted that small, 
naming-specific blockchains like Namecoin were vulnerable to 
51\% attacks~\cite{wei2017review}.

Prior work has evaluated the effectiveness of interventions 
that target DNS domains. Kesari et al. provide an overview of 
legal intervention methods and cites their use in a number of 
malware takedowns~\cite{kesari_deterring_2017}. Wang et al. 
studied the use of TROs to seize storefronts run by 
spammers~\cite{wang_search_2014}. Liu et al. analyzed the 
effectiveness of two interventions that were initiated by 
registrars and designed to stop 
spammers from registering 
domains~\cite{liu_registrar_intervention}. Prior literature 
has also analyzed interventions based on taking down hosting 
providers, and concluded that these interventions have modest 
or mixed effectiveness~\cite{bradbury2014testing, 
konte2015aswatch, noroozian_platforms_2019, alrwais_bph}. 