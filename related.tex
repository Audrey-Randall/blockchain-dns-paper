\section{Related Work}

% Kalodner 2015, An Empirical Study of Namecoin
Kalodner et al. performed the first study to our knowledge of 
Namecoin in 2015~\cite{kalodner_namecoin_2015}. They conclude 
that the Namecoin ecosystem was ``dysfunctional:'' only 28 
out of 120,000 registered names were valid, not squatted, and 
had nontrivial content.

% Unravelling Ariadne's Thread
Patsakis et al. present an analysis of potential weaknesses 
and user risks of 
Namecoin and Emercoin, including the risks of squatting, 51\% 
attacks, phishing, and abuse by 
malware~\cite{patsakis_unravelling_2020}. The authors also 
provide an overview of the names stored in these systems, and 
found that many names registered in the Alexa Top 1K were 
also registered under Namecoin and Emercoin's alt-TLDs. Most 
of these squatted names redirected to pornographic websites.

% Unearthing malicious campaigns and actors from the 
% blockchain DNS ecosystem~\cite{casino_unearthing_2021}, 
% good graphs.
Casino et al. analyzed the IP addresses in Namecoin and 
Emercoin records~\cite{casino_unearthing_2021}. They first 
identified malicious IP addresses using several threat 
intelligence databases, and then clustered all the IPs into 
``malicious,'' ``suspicious'' and ``benign'' categories with 
a ``poisoning'' 
approach. An IP was labeled ``malicious'' if a threat 
intelligence database categorized it as such. It was labeled 
``suspicious'' if it appeared in the same wallet, was 
resolved to by the same domain, or shared the same email TXT 
record as a malicious IP, and ``benign'' if it had no 
connection to a malicious IP. Casino et al. discovered that 
only 8\% of the IPs in Emercoin and 28\% of those in Namecoin 
had no association with malicious IP addresses. While this 
paper mentioned the existence of more recent blockchain 
naming systems, it did not perform an analysis of any system 
except Namecoin and Emercoin.

Do we need papers that have tried to make decentralized 
naming systems? 
Ariadne's Thread has a list.

Numerous other blockchain-based naming systems have been 
proposed, including
Blockstack~\cite{ali2016blockstack}, 
Bitforest/Conifer~\cite{dong2018bitforest, dong2018conifer}, 
BlockDNS~\cite{blockdns}