\section{Measurements of Name Resolution Queries}
\label{sec:b-root}

Because blockchain names require alternate resolution systems, we predicted 
that many of them would ``leak'' into the DNS when misconfigured machines 
attempt to resolve them as ordinary DNS domains. An observer might therefore be 
able to see which names are popular by measuring which blockchain names are 
requested through ordinary DNS requests. We took two 24-hour samples of names 
with Namecoin and Emercoin's alternate TLDs that appeared in 
queries to a root DNS server. One of these samples was from October 19, 2021, 
and the other was from April 15, 2022.

Discovering which names users are resolving in blockchain-based 
naming systems is challenging: reads within a 
blockchain system are not transactions, so they do not leave any evidence 
behind. We hypothesized that some attempts to resolve blockchain names 
would be misconfigured and might leak into regular DNS. To test this 
hypothesis, we obtained samples of the queries for domains with BNS alt-TLDs 
from one of the root DNS resolvers, B-root.

\subsection{Removing irrelevant queries}

Some queries for domains with alt-TLDs at B-root were unrelated to the 
blockchain systems they share alt-TLDs with. A prior study on root DNS queries 
found that some networks use non-ICANN TLDs internally, under the assumption 
that queries for those names will never reach external DNS resolvers. However, 
these queries frequently leak to external networks~\cite{chen_wpad_2016}. We 
observed that some internal networks use alt-TLDs that overlap with the 
blockchain naming systems' alt-TLDs. For example, we observed numerous queries 
for randomly generated subdomains of 
\texttt{kroger.com.btc}, \texttt{nordic.x}, \texttt{sata.x}, and others. These 
domains are not registered in any blockchain-based naming system. 
Distinguishing these names from 
blockchain-related names is particularly challenging when the names may be part 
of DGA-based malware campaigns. The obvious test for whether a name is meant to 
be blockchain-related is whether the name is registered in the blockchain 
naming system; however, when the name is a potential AGD, it may be 
unregistered even though it is blockchain-related. Unfortunately, we see a wide 
variety of potential AGDs across most alt-TLDs in the B-root sample.

We solved this problem by observing that we have some ground truth for what a 
malware campaign looks like from B-root's perspective. Many of the most 
frequently queried names at B-root are known to be 
associated with malware: an example of these names is shown in 
Table~\ref{tab:namecoin_emercoin}. These campaigns have two properties. 
First, the frequently queried names originate from multiple different ASes. 
Second, AGDs associated with malware campaigns fall into two categories on any 
given day: the names that are currently registered and resolve to the C2 
server, and the unregistered names. For each of the popular, randomly generated 
names in Table~\ref{tab:namecoin_emercoin}, we observe a large number of 
similar names, generated using the same algorithm, that have many fewer 
requests. We found that the randomly generated names with Namecoin and 
Emercoin's alt-TLDs had these properties indicative of a malware campaign, the 
randomly generated names using ENS and Unstoppable Domains' alt-TLDs did not. 
These randomly generated names usually only had one lookup 
each, and were universally unregistered: we never observed 
any names generated by the same DGA that were registered in a 
blockchain naming system. We concluded that these 
names were unlikely to be part of DGA-based malware 
campaigns. Therefore, we removed all names from the B-root 
sample that had ENS or Unstoppable Domains 
alt-TLDs but were unregistered in those naming systems. 



%We observed numerous names of the form <random\_string>.alt-TLD, where the 
%alt-TLD in question is used by ENS or Unstoppable Domains. These names 
%might have been indicative of malware using DGAs to create names for its C2 
%servers, because malware often sends queries for unregistered domains until it 
%discovers the correct domain for contacting its C2 server. However, on further 
%inspection, we predict 
%that these names are not related to malware for several 
%reasons. First, we observe that each randomly generated name is only requested 
%twice, and the requests originate from the same AS every time: one comes from 
%AS 15169 (Google) and the other comes from AS 394353 (B-root's own AS). We 
%suspect that this indicates one or two misbehaving machines, rather than a 
%malware campaign. Second, if the names were malware-related, periods 
%of time would exist when a particular name was registered, reachable, and used 
%to contact the C2 server. This name would presumably receive a high volume of 
%requests for that period of time, and no apparently randomly generated name in 
%our two-week sample ever received more than 22 requests (uk3hr2y7g.eth). 
%Finally, we compare these names to the randomly generated Emercoin and 
%Namecoin 
%names that were requested from B-root and known to be part of malware 
%campaigns. Examples of these names are given in 
%Table~\ref{tab:namecoin_emercoin}. The randomly 
%generated names from this group, such as those belonging to BazarLoader and 
%Necurs, fit our 
%predictions for how malware-related names behave: they receive far more 
%queries than the randomly generated names using ENS or Unstoppable alt-TLDs. 
%We 
%therefore removed potential ENS and Unstoppable Domains names from 
%the B-root sample if they were not registered in their respective naming 
%system.

We also discovered that domains used in the WPAD discovery protocol, a protocol 
used for local network participants to discover DNS proxies, used alt-TLDs that 
overlapped with blockchain DNS. WPAD DNS lookups take the form of 
\texttt{wpad.company-name.alt-TLD}~\cite{chen_wpad_2016}. If we observe a 
request of that form, we consider it to be strong evidence that other requests 
to subdomains of \texttt{company-name.alt-TLD} have escaped from an internal 
network. We remove these requests from our sample as well. 

\subsection{Queries for UD and ENS names}

%Second, we note that some of these names appear to be intended for typo 
%squatting. \randall{Tens of? Hundreds of?} lookups occur for variations of 
%popular names such as ``netflix,'' ``gmail,'' ``apple,'' and 
%``yahoo.'' The string ``netflix'' in particular was popular. \randall{All of 
	%these were .kred and 
	%.luxe and stuff, not Namecoin/Emercoin.
	
	
After removing names that are not registered in Unstoppable Domains, we 
observed no more than six names with each Unstoppable TLD requested from B-root 
on either of our sample days. Either no one is using Unstoppable to look up 
domains (consistent with the types of records they store), or Unstoppable has 
modest adoption but most people are using the extensions properly.

ENS names leaked slightly more frequently: we saw 186 names in the October 19th
sample and 313 in the April 15th sample.

\subsection{Queries for Namecoin and Emercoin Names}

\begin{table}
	\begin{tabular}{lrr}
		\toprule
		Malware & Domain & Lookups \\
		\midrule
		Gandcrab	&	malwarehunterteam.bit	&	696	\\
		&	politiaromana.bit	&	682	\\
		&	gdcb.bit	&	632	\\
		&	zonealarm.bit	&	1256	\\
		&	ransomware.bit	&	2078	\\
		&	nomoreransom.coin	&	2242	\\
		CHESSYLITE	&	leomoon.bit	&	1870	\\
		&	lookstat.bit	&	1420	\\
		&	sysmonitor.bit	&	1038	\\
		&	volstat.bit	&	910	\\
		&	xoonday.bit	&	1146	\\
		Dofoil	&	vrubl.bit	&	1976	\\
		&	levashov.bit	&	2118	\\
		&	vinik.bit	&	12530	\\
		KPOT Stealer	&	kpotuvorot10.bit	&	3902	\\
		&	star-fox.bit	&	701	\\
		Team9 Loader	&	bestgame.bazar	&	1884	\\
		&	forgame.bazar	&	1730	\\
		&	zirabuo.bazar	&	102	\\
		&	tallcareful.bazar	&	292	\\
		&	coastdeny.bazar	&	278	\\
		BazarLoader	&	acegikbcggin.bazar	&	1092	\\
		&	acegilbcggio.bazar	&	934	\\
		Trojan RTM	&	stat-counter-[0-9]-[0-9].bit	&	20996	\\
		Necurs	&	jfbbrj3bbbd.bit	&	3009	\\
		& qcmbartuop.bit & 2632 \\
		\bottomrule
	\end{tabular}
	\caption{Examples of malicious Namecoin and Emercoin domains in the October 
	sample of B-root 
		queries.}
	\label{tab:namecoin_emercoin}
\end{table}

%\begin{figure}[t]
%	\centering
%	
%\includegraphics[width=3in]{figs/namecoin_and_emercoin_tlds_num_names.pdf}
%	\caption{Number of unique Emercoin and Namecoin names 
%requested in day-long 
%		samples from B-root.}
%	\label{fig:namecoin_and_emercoin_names}
%\end{figure}
%
%\begin{figure}[t]
%	\centering
%	
%\includegraphics[width=3in]{figs/namecoin_and_emercoin_tlds_num_counts.pdf}
%	\caption{Total requests for Emercoin and Namecoin names 
%in day-long samples 
%		from B-root.}
%	\label{fig:namecoin_and_emercoin_counts}
%\end{figure}

The big takeaway is that there's a long tail which is mostly 
composed of AGDs that are each looked up once, and a small 
number of names (a few AGDs but mostly human-meaningful 
names) that are largely related to malware.

We examined the names from each sample 
and found ample evidence of names that are likely to be used by malware. We 
manually inspected 
the Emercoin and Namecoin names with the most queries, and found that the vast 
majority of the most 
frequently queried names were associated with malware. An example of malware 
names from the October 
sample of B-root queries is shown in Table~\ref{tab:namecoin_emercoin}. We 
found that many of the 
names appeared to be randomly generated, making them unlikely to be benign. 
\randall{not true, randomly generated could be non-malware we just don't know 
	what they are.} The only popular Emercoin 
or Namecoin name that was not associated with malware was 
\texttt{nnm-club.lib}, which appears to be a 
pirating site. These findings are consistent with previous work that found the 
vast majority of names 
and IPs on Emercoin and Namecoin to be associated with malicious 
behavior~\cite{patsakis_unravelling_2020, casino_unearthing_2021}. 
\randall{more results from these 
	papers?}


	
%\subsection{.x}
%
%When you remove the unregistered Unstoppable and ENS domains and the wpad 
%domains, there’s like ONE .x domain left. I guess a lot of the .x domains were 
%either misconfigured IP addresses (2218, found with 
%commented regex) or randomly generated names (6956, according to my heuristic 
%of no English words above either 4 or 5 chars (check this, probably redo this 
%number)) %([0-9]+|x+)\.([0-9]+|x+)\.([0-9]+|x+)\.x\.\s

Across all data from October and April, 39,849 .x names were misconfigured IP 
addresses.

%\subsection{.bit and .bazar}
%
%.bit names have a huge number of lookups for a small number of names. This is 
%in contrast to the randomly generated .eth and .x names, the vast majority of 
%which have two lookups each. Since we are pretty sure the majority of .bit 
%names are malware-related, and having tons of lookups is consistent with how 
%DGAs work, we predict that the randomly generated .x, .eth, and .bitcoin 
%domains are not attempts to find C2 domains.


\subsection{Frequency of requests}

We investigated how many days each name was requested, and found that the vast 
majority of names are only requested once, on a single day. There were two 
notable exceptions: a group of \texttt{.bit} names that were requested 
on every day of the sample, and a group of \texttt{.bazar} names that 
were requested on most or all days of the sample.

We analyzed the group of .bit names that were requested on all 14 days of our 
sample and had more total requests than any name requested on fewer than 14 
days. 66 names fit this criteria. We submitted them to VirusTotal and found 
that only 18 names were not labeled malicious by any engine, while 48 were 
labeled malicious by at least one. 

The \texttt{.bazar} names that were requested on most (>10) days for our sample 
fell into two categories. The first 
contained AGDs that appear to be generated by concatenating four 
lowercase-letter bigrams consisting of a consonant and a vowel (e.g., 
acbaelek.bazar, acbaelel.bazar, acbaelid.bazar). These names appear to be 
generated by the 
malware BazarLoader~\cite{bazarloader_dga}. The second category contains 38 
names that do not appear to be randomly generated. We uploaded these to 
VirusTotal and determined that 23 were labeled as malicious by at least one 
threat intelligence service, three were not indexed by VirusTotal, and the 
remainder were not labeled as malicious. Six of the names were themed around 
Australian tourism, of which four were labeled malicious and two were not: 
these names may also be associated with 
BazarLoader~\cite{alienvault_bazarloader}. 


%\begin{figure}[t]
%	\centering
%	
%\includegraphics[width=3in]{figs/ud_and_ens_tlds_num_names_no_wpad_or_unregistered_ens_ud.pdf}
%	\caption{Number of unique Emercoin and Namecoin names requested in day-long 
	%		samples from B-root.}
%	\label{fig:ud_and_ens_names}
%\end{figure}
%\begin{figure}[t]
%	\centering
%	
%\includegraphics[width=3in]{figs/ud_and_ens_tlds_num_counts_no_wpad_or_unregistered_ens_ud.pdf}
%	\caption{Total requests for Emercoin and Namecoin names in day-long samples 
	%		from B-root.}
%	\label{fig:ud_and_ens_counts}
%\end{figure}