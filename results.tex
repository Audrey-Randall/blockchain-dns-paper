\section{Measurements of Name Resolution Queries}
\label{sec:b-root}
% this is the b-root section

Because blockchain names require alternate resolution systems, we predicted 
that many of them would ``leak'' into the DNS when misconfigured machines 
attempt to resolve them as ordinary DNS domains. An observer might therefore be 
able to see which names are popular by measuring which blockchain names are 
requested through ordinary DNS requests. We took two 24-hour samples of names 
with Namecoin and Emercoin's alternate TLDs that appeared in 
queries to a root DNS server. One of these samples was from October 19, 2021, 
and the other was from April 15, 2022.

Discovering which names users are resolving in blockchain-based 
naming systems is challenging: methods that work in traditional DNS, such as 
cache snooping, rely on knowing which domains to look for, and reads within a 
blockchain system don't leave traces in cache the way lookups to ordinary 
resolvers do. We hypothesized that some attempts to resolve blockchain names 
would be misconfigured and might leak into regular DNS. To test this 
hypothesis, we obtained samples of the queries for domains with BNS alt-TLDs 
from one of the root DNS resolvers, B-root.

\subsection{Removing irrelevant queries}

Some queries for domains with alt-TLDs at B-root were unrelated to the 
blockchain systems they share alt-TLDs with. A prior study on root DNS queries 
found that some networks use non-ICANN TLDs internally under the assumption 
that queries for those names will never reach external DNS resolvers. However, 
these queries frequently leak to external networks~\cite{chen_wpad_2016}. We 
observed that some 
internal networks use alt-TLDs that overlap with the blockchain naming systems' 
alt-TLDs. For example, we observed numerous queries for subdomains of 
\texttt{kroger.com.btc}, \texttt{nordic.x}, \texttt{sata.x}, and others. These 
domains are not registered in the Unstoppable Domains namespace. We removed 
names from the B-root sample if they were not registered in their respective 
blockchain-based naming system.

We also discovered that domains used in the WPAD discovery protocol, a protocol 
used for local network participants to discover DNS proxies, used alt-TLDs that 
overlapped with blockchain DNS. WPAD DNS lookups take the form of 
\texttt{wpad.company-name.alt-TLD}~\cite{chen_wpad_2016}. If we observe a 
request of that form, we consider it to be strong evidence that other requests 
to subdomains of \texttt{company-name.alt-TLD} have escaped from an internal 
network. We remove these requests from our sample as well.

After removing names that are not registered in Unstoppable Domains, we 
observed no more than six names with each Unstoppable TLD requested from B-root 
on either of our sample days. Either no one is using Unstoppable to look up 
domains (consistent with the types of records they store), or Unstoppable has 
modest adoption but most people are using the extensions properly.

ENS names leaked slightly more frequently: we saw 186 names in the October 19th
sample and 313 in the April 15th sample. 

But what about the unregistered domains?

\subsection{Queries for Namecoin and Emercoin Names}

\begin{table}
	\begin{tabular}{lrr}
		\toprule
		Malware & Domain & Lookups \\
		\midrule
		Gandcrab	&	malwarehunterteam.bit	&	696	\\
		&	politiaromana.bit	&	682	\\
		&	gdcb.bit	&	632	\\
		&	zonealarm.bit	&	1256	\\
		&	ransomware.bit	&	2078	\\
		&	nomoreransom.coin	&	2242	\\
		CHESSYLITE	&	leomoon.bit	&	1870	\\
		&	lookstat.bit	&	1420	\\
		&	sysmonitor.bit	&	1038	\\
		&	volstat.bit	&	910	\\
		&	xoonday.bit	&	1146	\\
		Dofoil	&	vrubl.bit	&	1976	\\
		&	levashov.bit	&	2118	\\
		&	vinik.bit	&	12530	\\
		KPOT Stealer	&	kpotuvorot10.bit	&	3902	\\
		&	star-fox.bit	&	701	\\
		Team9 Loader	&	bestgame.bazar	&	1884	\\
		&	forgame.bazar	&	1730	\\
		&	zirabuo.bazar	&	102	\\
		&	tallcareful.bazar	&	292	\\
		&	coastdeny.bazar	&	278	\\
		BazarLoader	&	acegikbcggin.bazar	&	1092	\\
		&	acegilbcggio.bazar	&	934	\\
		Trojan RTM	&	stat-counter-[0-9]-[0-9].bit	&	20996	\\
		Necurs	&	jfbbrj3bbbd.bit	&	3009	\\
		\bottomrule
	\end{tabular}
	\caption{Examples of malicious Namecoin and Emercoin domains in the October 
	sample of B-root 
		queries.}
	\label{tab:namecoin_emercoin}
\end{table}

\begin{figure}[t]
	\centering
	\includegraphics[width=3in]{figs/namecoin_and_emercoin_tlds_num_names.pdf}
	\caption{Number of unique Emercoin and Namecoin names requested in day-long 
		samples from B-root.}
	\label{fig:namecoin_and_emercoin_names}
\end{figure}

\begin{figure}[t]
	\centering
	\includegraphics[width=3in]{figs/namecoin_and_emercoin_tlds_num_counts.pdf}
	\caption{Total requests for Emercoin and Namecoin names in day-long samples 
		from B-root.}
	\label{fig:namecoin_and_emercoin_counts}
\end{figure}

 We examined the names from each sample 
and found ample evidence of names that are likely to be used by malware. We 
manually inspected 
the Emercoin and Namecoin names with the most queries, and found that the vast 
majority of the most 
frequently queried names were associated with malware. An example of malware 
names from the October 
sample of B-root queries is shown in Table~\ref{tab:namecoin_emercoin}. We 
found that many of the 
names appeared to be randomly generated, making them unlikely to be benign. 
\randall{not true, randomly generated could be non-malware we just don't know 
	what they are.} The only popular Emercoin 
or Namecoin name that was not associated with malware was 
\texttt{nnm-club.lib}, which appears to be a 
pirating site. These findings are consistent with previous work that found the 
vast majority of names 
and IPs on Emercoin and Namecoin to be associated with malicious 
behavior~\cite{patsakis_unravelling_2020, casino_unearthing_2021}. 
\randall{more results from these 
	papers?}

\subsection{Queries for UD and ENS names}

%Second, we note that some of these names appear to be intended for typo 
%squatting. \randall{Tens of? Hundreds of?} lookups occur for variations of 
%popular names such as ``netflix,'' ``gmail,'' ``apple,'' and 
%``yahoo.'' The string ``netflix'' in particular was popular. \randall{All of 
%these were .kred and 
	%.luxe and stuff, not Namecoin/Emercoin.
	
\subsection{.x}

When you remove the unregistered Unstoppable and ENS domains and the wpad 
domains, there’s like ONE .x domain left. I guess a lot of the .x domains were 
either misconfigured IP addresses (2218, found with 
commented regex) or randomly generated names (6956, according to my heuristic 
of no English words above either 4 or 5 chars (check this, probably redo this 
number)) %([0-9]+|x+)\.([0-9]+|x+)\.([0-9]+|x+)\.x\.\s

\subsection{.bit and .bazar}

.bit names have a huge number of lookups for a small number of names. This is 
in contrast to the randomly generated .eth and .x names, the vast majority of 
which have two lookups each. Since we are pretty sure the majority of .bit 
names are malware-related, and having tons of lookups is consistent with how 
DGAs work, we predict that the randomly generated .x, .eth, and .bitcoin 
domains are not attempts to find C2 domains.



%\begin{figure}[t]
%	\centering
%	
%\includegraphics[width=3in]{figs/ud_and_ens_tlds_num_names_no_wpad_or_unregistered_ens_ud.pdf}
%	\caption{Number of unique Emercoin and Namecoin names requested in day-long 
	%		samples from B-root.}
%	\label{fig:ud_and_ens_names}
%\end{figure}
%\begin{figure}[t]
%	\centering
%	
%\includegraphics[width=3in]{figs/ud_and_ens_tlds_num_counts_no_wpad_or_unregistered_ens_ud.pdf}
%	\caption{Total requests for Emercoin and Namecoin names in day-long samples 
	%		from B-root.}
%	\label{fig:ud_and_ens_counts}
%\end{figure}