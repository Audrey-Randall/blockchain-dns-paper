\section{Measurements of Name Resolution Queries}
\label{sec:b-root}

\begin{table}
	% These numbers have been cut in half to account for responses
	\begin{tabular}{lrrr}
		\toprule
		Malware & Domain & Lookups & Source\\
		\midrule
		Gandcrab	&	malwarehunterteam.bit	&	348	& 
		\cite{trellix_gandcrab} \\
		&	politiaromana.bit	&	341	& \cite{trellix_gandcrab} \\
		&	gdcb.bit	&	316	& \cite{trellix_gandcrab} \\
		&	zonealarm.bit	&	628	& \cite{acronis_gandcrab} \\
		&	ransomware.bit	&	1,039 & \cite{acronis_gandcrab} \\
%		&	nomoreransom.coin	&	1121	\\
		CHESSYLITE	&	leomoon.bit	&	935	& \cite{mandiant_chessylite} \\
		&	lookstat.bit	&	710	& \cite{mandiant_chessylite} \\
		&	sysmonitor.bit	&	519	& \cite{mandiant_chessylite} \\
		&	volstat.bit	&	455	& \cite{mandiant_chessylite} \\
		&	xoonday.bit	&	573	& \cite{mandiant_chessylite} \\
		Dofoil	&	vrubl.bit	&	988	& \cite{dofoil_2018} \\
		&	levashov.bit	&	1,059 & \cite{dofoil_2018} \\
		&	vinik.bit	&	6,265 & \cite{dofoil_2018} \\
		KPOT Stealer	&	kpotuvorot10.bit	&	1,951 & \cite{kpot_2021}\\
		&	star-fox.bit	&	351	 & \cite{joesandbox} \\
		Team9 Loader	&	bestgame.bazar	&	942	& \cite{team9_fox-it} \\
		&	forgame.bazar	&	865	& \cite{team9_fox-it} \\
		&	zirabuo.bazar	&	51	& \cite{team9_fox-it} \\
		&	tallcareful.bazar	&	146 & \cite{team9_fox-it} 	\\
		&	coastdeny.bazar	&	139	& \cite{team9_fox-it} \\
		BazarLoader	&	acegikbcggin.bazar	&	546	& \cite{baza_proofpoint} \\
		&	acegilbcggio.bazar	&	467	& \cite{baza_proofpoint} \\
		Trojan RTM	&	stat-counter-[0-9]-[0-9].bit & 10,498 & 
		\cite{sudonull_trojanrtm} \\
		Necurs	&	jfbbrj3bbbd.bit	&	1,505 & \cite{threatcrowd_necurs}\\
		& qcmbartuop.bit & 1,316 & \cite{dgas_of_necurs} \\
		\bottomrule
	\end{tabular}
	\caption{Examples of malicious Namecoin and Emercoin domains in the October 
		sample of B-root 
		queries.}
	\label{tab:namecoin_emercoin}
\end{table}

Our findings up until now have indicated that
malware still heavily utilizes Emercoin and Namecoin, but has 
not made the jump to ENS and Unstoppable Domains. However, to fully test this 
assumption, we needed to check not only which domains were \emph{registered} in 
each system, but also which domains were being \emph{used}. 
This is challenging because name resolutions are not 
transactions: they are read-only operations that do not leave 
a record in the blockchain. We cannot directly measure 
usage of blockchain names, but we observed that a side 
channel might exist to estimate name usage: ``leakage'' to the DNS. We 
predicted that since blockchain names require configuring alternate resolution 
systems, some might ``leak'' into the DNS when misconfigured machines 
attempt to resolve them as ordinary DNS domains. These leaked 
names would be visible at the root DNS servers, but would not 
be forwarded to any other DNS servers, because the roots 
would respond that the alt-TLDs do not exist. An observer who could see which 
names were requested at a DNS root server with alt-TLDs corresponding to 
blockchain naming systems could get a sense for which names are in use.

We therefore took two samples of the names that were 
requested at B-root over the course of several days. The first sample consisted 
of names and how many requests were made for each on October 19, 2021. 
The second sample consisted of names, numbers of requests, and the ASes the 
requests were made from. It spanned two weeks in April 2022, from April 16 to 
April 30. 

%
Another advantage of using B-root as a vantage point was 
that it let us observe requests for \emph{unregistered} AGDs that 
might indicate the presence of malware. DGAs work by generating a vast number 
of names, but only a few are ever registered and functional at any given time. 
These unregistered names do not, of course, appear in our samples of the 
registered names in each blockchain naming system. An infected host determines 
which names are registered by simply attempting to resolve them. If an infected 
host's queries were leaking into the DNS, we theorized that these queries would 
be very obvious, because the infected host would always receive NXDOMAIN 
responses from the root. These responses would cause the infected host to 
assume it has not 
found the correct C2 name for today and keep trying new 
names. The flood of nonexistent names with blockchain-based 
alt-TLDs would be visible when examining queries arriving at B-root. 

\subsection{Frequently Accessed Names}

We first investigated how many days each name was requested on, and found that 
the vast majority of names are only requested once, on a single day. There were 
two notable exceptions: a small group (67) of \texttt{.bit} names that received 
a high volume of requests on every day of the sample, and a large group 
(~39,000) of unique \texttt{.bazar} names that 
were each requested on most or all days of the sample.

We analyzed the group of .bit names that were requested on all 14 days of our 
sample and had more total requests than any name requested on fewer than 14 
days. 66 names fit this criteria. We submitted them to VirusTotal and found 
that only 18 names were not labeled malicious by any engine, while 48 were 
labeled malicious by at least one. 

The \texttt{.bazar} names that were requested on most (>10) days for our sample 
fell into two categories. The first 
contained AGDs that appear to be generated by concatenating four 
lowercase-letter bigrams consisting of a consonant and a vowel (e.g., 
\texttt{acbaelek.bazar}, \texttt{acbaelel.bazar}, \texttt{acbaelid.bazar}). 
These names appear to be 
generated by the 
malware BazarLoader~\cite{bazarloader_dga}. The second category contains 38 
names that do not appear to be randomly generated. We uploaded these to 
VirusTotal and determined that 23 were labeled as malicious by at least one 
threat intelligence service, three were not indexed by VirusTotal, and the 
remainder were not labeled as malicious. Six of the names were themed around 
Australian tourism, of which four were labeled malicious and two were not: 
these names may also be associated with 
BazarLoader~\cite{alienvault_bazarloader}. 

We note that the most popular Emercoin and Namecoin names each day were largely 
known to be associated with malware, which we determined by manually searching 
the Internet. We present a sample of the most popular malware-related names in 
Table~\ref{tab:namecoin_emercoin}. These names were 
taken from the October sample; the days in April had a similarly high number of 
malicious names that received high volumes of requests. 

\subsection{Unregistered ENS and Unstoppable Domains names}
 
We observed a large number of names, mostly AGDs, with alt-TLDs that are used 
by ENS and Unstoppable Domains. However, we found that these names are actually 
unrelated to blockchain naming systems and are likely not part of malware 
campaigns. We drew this conclusion for two reasons. First, the randomly 
generated names
only had one lookup each, and all of these lookups originated from a single AS 
(AS15169, Google). This is in contrast to lookups for AGDs in Emercoin and 
Namecoin that are known to be part of malware campaigns, whose lookups 
originate from many different ASes and where some names receive many more than 
a single lookup request. Second, not a single AGD with an ENS or Unstoppable 
Domains alt-TLD was registered in a blockchain naming system. If these names 
had been part of a malware campaign, at least one should have resolved to the 
address of a C2 server at some point. It is possible that B-root only received 
failed requests from a single misconfigured machine, but this does not match 
the behavior we observe for malware campaigns that abuse Namecoin and Emercoin.

We predict that rather than being intended for use in a blockchain naming 
system, these AGDs were leaked from local networks, and were never intended to 
be resolved by either a blockchain naming system \emph{or} the DNS root. A 
prior study on root DNS queries 
found that some networks use non-ICANN TLDs internally, under the assumption 
that queries for those names will never reach external DNS resolvers. However, 
these queries frequently leak to external networks~\cite{chen_wpad_2016}. We 
predict that some internal networks use alt-TLDs that coincidentally overlap 
with the blockchain naming systems' alt-TLDs. We concluded that these names 
were unlikely to be part of DGA-based malware campaigns, and were also likely 
unrelated to blockchain naming systems at all.

\subsection{Registered names from ENS and Unstoppable Domains}

Very few registered names from ENS or Unstoppable Domains leaked to B-root: we 
observed 
fewer than 400 unique ENS names per day and fewer than 300 unique daily names 
from Unstoppable Domains. These names also received few requests per day 
compared to the names from Namecoin and Emercoin. No name received more than 
approximately 350 lookups, % this number has been split in half already
in contrast with the most popular domains in 
Namecoin, which received an order of magnitude more requests per day. 
We submitted every ENS and Unstoppable Domains name that received more than ten 
daily requests to VirusTotal. None were in VirusTotal's database, in contrast 
with names from Emercoin and Namecoin, which were largely present and flagged 
as malicious.

Each of these findings regarding names that leak to B-root support our 
conclusion that malware still heavily utilizes older systems like Namecoin and 
Emercoin, but has not yet adopted new systems like ENS or Unstoppable Domains. 


%Some queries for domains with alt-TLDs at B-root were unrelated to the 
%blockchain systems they share alt-TLDs with. A prior study on root DNS queries 
%found that some networks use non-ICANN TLDs internally, under the assumption 
%that queries for those names will never reach external DNS resolvers. However, 
%these queries frequently leak to external networks~\cite{chen_wpad_2016}. We 
%observed that some internal networks use alt-TLDs that overlap with the 
%blockchain naming systems' alt-TLDs. For example, we observed numerous queries 
%for randomly generated subdomains of 
%\texttt{kroger.com.btc}, \texttt{nordic.x}, \texttt{sata.x}, and others. These 
%domains are not registered in any blockchain-based naming system. 
%Distinguishing these names from 
%blockchain-related names is particularly challenging when the names may be 
%part 
%of DGA-based malware campaigns. The obvious test for whether a name is meant 
%to 
%be blockchain-related is whether the name is registered in the blockchain 
%naming system; however, when the name is a potential AGD, it may be 
%unregistered even though it is blockchain-related. Unfortunately, we see a 
%wide 
%variety of potential AGDs across most alt-TLDs in the B-root sample.
%
%We solved this problem by observing that we have some ground truth for what a 
%malware campaign looks like from B-root's perspective. Many of the most 
%frequently queried names at B-root are known to be 
%associated with malware: an example of these names is shown in 
%Table~\ref{tab:namecoin_emercoin}. These campaigns have two properties. 
%First, the frequently queried names originate from many different ASes. 
%Second, AGDs associated with malware campaigns fall into two categories on any 
%given day: the names that are currently registered and resolve to the C2 
%server, and the unregistered names. For each of the popular, randomly 
%generated 
%names in Table~\ref{tab:namecoin_emercoin}, we observe a large number of 
%similar names, generated using the same algorithm, that have many fewer 
%requests. We found that the randomly generated names with Namecoin and 
%Emercoin's alt-TLDs had these properties indicative of a malware campaign, the 
%randomly generated names using ENS and Unstoppable Domains' alt-TLDs did not. 




%We observed numerous names of the form <random\_string>.alt-TLD, where the 
%alt-TLD in question is used by ENS or Unstoppable Domains. These names 
%might have been indicative of malware using DGAs to create names for its C2 
%servers, because malware often sends queries for unregistered domains until it 
%discovers the correct domain for contacting its C2 server. However, on further 
%inspection, we predict 
%that these names are not related to malware for several 
%reasons. First, we observe that each randomly generated name is only requested 
%twice, and the requests originate from the same AS every time: one comes from 
%AS 15169 (Google) and the other comes from AS 394353 (B-root's own AS). We 
%suspect that this indicates one or two misbehaving machines, rather than a 
%malware campaign. Second, if the names were malware-related, periods 
%of time would exist when a particular name was registered, reachable, and used 
%to contact the C2 server. This name would presumably receive a high volume of 
%requests for that period of time, and no apparently randomly generated name in 
%our two-week sample ever received more than 22 requests (uk3hr2y7g.eth). 
%Finally, we compare these names to the randomly generated Emercoin and 
%Namecoin 
%names that were requested from B-root and known to be part of malware 
%campaigns. Examples of these names are given in 
%Table~\ref{tab:namecoin_emercoin}. The randomly 
%generated names from this group, such as those belonging to BazarLoader and 
%Necurs, fit our 
%predictions for how malware-related names behave: they receive far more 
%queries than the randomly generated names using ENS or Unstoppable alt-TLDs. 
%We 
%therefore removed potential ENS and Unstoppable Domains names from 
%the B-root sample if they were not registered in their respective naming 
%system.

%We also discovered that domains used in the WPAD discovery protocol, a 
%protocol 
%used for local network participants to discover DNS proxies, used alt-TLDs 
%that 
%overlapped with blockchain DNS. WPAD DNS lookups take the form of 
%\texttt{wpad.company-name.alt-TLD}~\cite{chen_wpad_2016}. If we observe a 
%request of that form, we consider it to be strong evidence that other requests 
%to subdomains of \texttt{company-name.alt-TLD} have escaped from an internal 
%network. We remove these requests from our sample as well. 


%\subsection{Queries for Namecoin and Emercoin Names}



%\begin{figure}[t]
%	\centering
%	
%\includegraphics[width=3in]{figs/namecoin_and_emercoin_tlds_num_names.pdf}
%	\caption{Number of unique Emercoin and Namecoin names 
%requested in day-long 
%		samples from B-root.}
%	\label{fig:namecoin_and_emercoin_names}
%\end{figure}
%
%\begin{figure}[t]
%	\centering
%	
%\includegraphics[width=3in]{figs/namecoin_and_emercoin_tlds_num_counts.pdf}
%	\caption{Total requests for Emercoin and Namecoin names 
%in day-long samples 
%		from B-root.}
%	\label{fig:namecoin_and_emercoin_counts}
%\end{figure}

%The big takeaway is that there's a long tail which is mostly 
%composed of AGDs that are each looked up once, and a small 
%number of names (a few AGDs but mostly human-meaningful 
%names) that are largely related to malware.
%
%We examined the names from each sample 
%and found ample evidence of names that are likely to be used by malware. We 
%manually inspected 
%the Emercoin and Namecoin names with the most queries, and found that the vast 
%majority of the most 
%frequently queried names were associated with malware. An example of malware 
%names from the October 
%sample of B-root queries is shown in Table~\ref{tab:namecoin_emercoin}. We 
%found that many of the 
%names appeared to be randomly generated, making them unlikely to be benign. 
%\randall{not true, randomly generated could be non-malware we just don't know 
%	what they are.} The only popular Emercoin 
%or Namecoin name that was not associated with malware was 
%\texttt{nnm-club.lib}, which appears to be a 
%pirating site. These findings are consistent with previous work that found the 
%vast majority of names 
%and IPs on Emercoin and Namecoin to be associated with malicious 
%behavior~\cite{patsakis_unravelling_2020, casino_unearthing_2021}. 
%\randall{more results from these 
%	papers?}


	
%\subsection{.x}
%
%When you remove the unregistered Unstoppable and ENS domains and the wpad 
%domains, there’s like ONE .x domain left. I guess a lot of the .x domains were 
%either misconfigured IP addresses (2218, found with 
%commented regex) or randomly generated names (6956, according to my heuristic 
%of no English words above either 4 or 5 chars (check this, probably redo this 
%number)) %([0-9]+|x+)\.([0-9]+|x+)\.([0-9]+|x+)\.x\.\s
%
%Across all data from October and April, 39,849 .x names were misconfigured IP 
%addresses.

%\subsection{.bit and .bazar}
%
%.bit names have a huge number of lookups for a small number of names. This is 
%in contrast to the randomly generated .eth and .x names, the vast majority of 
%which have two lookups each. Since we are pretty sure the majority of .bit 
%names are malware-related, and having tons of lookups is consistent with how 
%DGAs work, we predict that the randomly generated .x, .eth, and .bitcoin 
%domains are not attempts to find C2 domains.



%\begin{figure}[t]
%	\centering
%	
%\includegraphics[width=3in]{figs/ud_and_ens_tlds_num_names_no_wpad_or_unregistered_ens_ud.pdf}
%	\caption{Number of unique Emercoin and Namecoin names requested in day-long 
	%		samples from B-root.}
%	\label{fig:ud_and_ens_names}
%\end{figure}
%\begin{figure}[t]
%	\centering
%	
%\includegraphics[width=3in]{figs/ud_and_ens_tlds_num_counts_no_wpad_or_unregistered_ens_ud.pdf}
%	\caption{Total requests for Emercoin and Namecoin names in day-long samples 
	%		from B-root.}
%	\label{fig:ud_and_ens_counts}
%\end{figure}