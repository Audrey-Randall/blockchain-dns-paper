\section{Measurements of Name Resolution Queries}
\label{sec:b-root}

\begin{table}
	% These numbers have been cut in half to account for responses
	\begin{tabular}{lrrr}
		\toprule
		Malware & Domain & Lookups & Source\\
		\midrule
		Gandcrab	&	malwarehunterteam.bit	&	348	& 
		\cite{trellix_gandcrab} \\
		&	politiaromana.bit	&	341	& \cite{trellix_gandcrab} \\
		&	gdcb.bit	&	316	& \cite{trellix_gandcrab} \\
		&	zonealarm.bit	&	628	& \cite{acronis_gandcrab} \\
		&	ransomware.bit	&	1,039 & \cite{acronis_gandcrab} \\
%		&	nomoreransom.coin	&	1121	\\
		CHESSYLITE	&	leomoon.bit	&	935	& \cite{mandiant_chessylite} \\
		&	lookstat.bit	&	710	& \cite{mandiant_chessylite} \\
		&	sysmonitor.bit	&	519	& \cite{mandiant_chessylite} \\
		&	volstat.bit	&	455	& \cite{mandiant_chessylite} \\
		&	xoonday.bit	&	573	& \cite{mandiant_chessylite} \\
		Dofoil	&	vrubl.bit	&	988	& \cite{dofoil_2018} \\
		&	levashov.bit	&	1,059 & \cite{dofoil_2018} \\
		&	vinik.bit	&	6,265 & \cite{dofoil_2018} \\
		KPOT Stealer	&	kpotuvorot10.bit	&	1,951 & \cite{kpot_2021}\\
		&	star-fox.bit	&	351	 & \cite{joesandbox} \\
		Team9 Loader	&	bestgame.bazar	&	942	& \cite{team9_fox-it} \\
		&	forgame.bazar	&	865	& \cite{team9_fox-it} \\
		&	zirabuo.bazar	&	51	& \cite{team9_fox-it} \\
		&	tallcareful.bazar	&	146 & \cite{team9_fox-it} 	\\
		&	coastdeny.bazar	&	139	& \cite{team9_fox-it} \\
		BazarLoader	&	acegikbcggin.bazar	&	546	& \cite{baza_proofpoint} \\
		&	acegilbcggio.bazar	&	467	& \cite{baza_proofpoint} \\
		Trojan RTM	&	stat-counter-[0-9]-[0-9].bit & 10,498 & 
		\cite{sudonull_trojanrtm} \\
		Necurs	&	jfbbrj3bbbd.bit	&	1,505 & \cite{threatcrowd_necurs}\\
		& qcmbartuop.bit & 1,316 & \cite{dgas_of_necurs} \\
		\bottomrule
	\end{tabular}
	\caption{Examples of malicious Namecoin and Emercoin domains in the October 
		sample of B-root 
		queries.}
	\label{tab:namecoin_emercoin}
\end{table}

Our analysis of the registered names in each blockchain naming system 
(Section~\ref{sec:naming_systems}) indicated that malware is not yet utilizing 
ENS and Unstoppable Domains. In contrast, recent work on the 
records stored in 
Emercoin and Namecoin found that these systems were heavily 
used by malware as recently as 2020 
~\cite{casino_unearthing_2021}. 
However, to test whether malware is still using Namecoin and Emercoin and is 
not using ENS and Unstoppable Domains, it is necessary to 
analyze not only which names are 
\emph{registered} in each system, but also which ones are 
heavily
\emph{used}.
This is challenging because name resolutions are not 
transactions: they are read-only operations that do not leave 
a record on the blockchain. We cannot directly measure 
usage of blockchain names, but we observed that a side 
channel might exist to estimate name usage: ``leakage'' to the DNS. We 
predicted that since blockchain names require configuring alternate resolution 
systems, some requests might ``leak'' into the DNS when misconfigured machines 
attempt to resolve them as ordinary DNS domains. These leaked 
names would be visible at the root DNS servers, but would not 
be forwarded to any other DNS servers, because the roots 
would respond that the alt-TLDs do not exist. An observer who could see which 
names were requested at a DNS root server with alt-TLDs corresponding to 
blockchain naming systems could get a sense for which names are in use.

We therefore took two samples of the names that were requested at the DNS 
B-root servers over the course of several days. The first 
sample consisted 
of names and how many requests were made for each on October 19, 2021. 
The second sample consisted of names, numbers of requests, and the ASes the 
requests were made from. It spanned two weeks in April 2022, from April 16 to 
April 30. 

%
Another advantage of using B-root as a vantage point was 
that it let us observe requests for \emph{unregistered} names 
that might indicate the presence of malware. DGAs work by 
generating a vast number 
of names, but only a few are ever registered and functional at any given time. 
These unregistered names do not, of course, appear in our samples of the 
registered names in each blockchain naming system. An infected host determines 
which names are registered by simply attempting to resolve them. If an infected 
host's queries were leaking into the DNS, we theorized that these queries would 
be very obvious, because the infected host would always receive NXDOMAIN 
responses from the root. These responses would cause the infected host to 
assume it has not 
found the correct C2 name for today and keep trying new 
names. The flood of nonexistent names with blockchain-based 
alt-TLDs would be visible when examining queries arriving at B-root. 

\subsection{Frequently Accessed Names}

We first investigated how many days each name was requested on, and found that 
the vast majority of names are only requested once, on a single day. There were 
two notable exceptions: a small group (67) of \texttt{.bit} names that received 
a high volume of requests on every day of the sample, and a large group 
(~39,000) of unique \texttt{.bazar} names that 
were each requested on most or all days of the sample. These 
names belong to the Namecoin and Emercoin naming systems, 
respectively.

We analyzed the group of \texttt{.bit} names that were 
requested on all 14 days of our 
sample and had more total requests than any name requested on fewer than 14 
days. 66 names fit this criteria. We submitted them to VirusTotal and found 
that only 18 names were not labeled malicious by any engine, while 48 were 
labeled malicious by at least one. 

The \texttt{.bazar} names that were requested on most (more 
than 10) 
days for our sample 
fell into two categories. The first 
contained names that appear to be generated by concatenating 
four 
lowercase-letter bigrams consisting of a consonant and a vowel (e.g., 
\texttt{acbaelek.bazar}, \texttt{acbaelel.bazar}, \texttt{acbaelid.bazar}). 
These names appear to be 
generated by the 
malware BazarLoader~\cite{bazarloader_dga}. The second category contains 38 
names that do not appear to be randomly generated. We uploaded these to 
VirusTotal and determined that 23 were labeled as malicious by at least one 
threat intelligence service, three were not indexed by VirusTotal, and the 
remainder were not labeled as malicious. Six of the names were themed around 
Australian tourism, of which four were labeled malicious and two were not: 
these names may also be associated with 
BazarLoader~\cite{alienvault_bazarloader}. 

We note that the most popular Emercoin and Namecoin names each day were largely 
known to be associated with malware, which we determined by manually searching 
the Internet. We present a sample of the most popular malware-related names in 
Table~\ref{tab:namecoin_emercoin}. These names were 
taken from the October sample; the days in April had a similarly high number of 
malicious names that received high volumes of requests. 

\subsection{Unregistered ENS and Unstoppable Domains names}
 
We observed a large number of names, mostly randomly 
generated names, with alt-TLDs that are used 
by ENS and Unstoppable Domains. However, we found that these names are actually 
unrelated to blockchain naming systems and are likely not part of malware 
campaigns. We drew this conclusion for two reasons. First, the randomly 
generated names
only had one lookup each, and all of these lookups originated from a single AS 
(AS15169, Google). This is in contrast to lookups for 
randomly generated names in Emercoin and 
Namecoin that are known to be part of malware campaigns: 
these requests 
originate from many different ASes and some names receive 
many more than just one request. Second, not a single 
randomly generated domain with an ENS or Unstoppable 
Domains alt-TLD was registered in a blockchain naming system. If these names 
had been part of a malware campaign, at least one should have resolved to the 
address of a C2 server at some point. It is possible that B-root only received 
failed requests from a single misconfigured machine, but this does not match 
the behavior we observe for malware campaigns that abuse Namecoin and Emercoin.

We predict that rather than being intended for use in a blockchain naming 
system, these randomly generated names were leaked from local 
networks, and were never intended to 
be resolved by either a blockchain naming system \emph{or} the DNS root. A 
prior study on root DNS queries 
found that some networks use non-ICANN TLDs internally, under the assumption 
that queries for those names will never reach external DNS resolvers. However, 
these queries frequently leak to external networks~\cite{chen_wpad_2016}. We 
predict that some internal networks use alt-TLDs that coincidentally overlap 
with the blockchain naming systems' alt-TLDs. We concluded that these names 
were unlikely to be part of DGA-based malware campaigns, and were also likely 
unrelated to blockchain naming systems at all.

\subsection{Requests for registered names from ENS and 
Unstoppable Domains}

Very few registered names from ENS or Unstoppable Domains leaked to B-root: we 
observed 
fewer than 400 unique ENS names per day and fewer than 300 unique daily names 
from Unstoppable Domains. These names also received few requests per day 
compared to the names from Namecoin and Emercoin. No name received more than 
approximately 350 lookups, % this number has been split in half already
in contrast with the most popular domains in 
Namecoin, which received an order of magnitude more requests per day. 
We submitted every ENS and Unstoppable Domains name that received more than ten 
daily requests to VirusTotal. None were in VirusTotal's database, in contrast 
with names from Emercoin and Namecoin, which were largely present and flagged 
as malicious.

Each of these findings regarding names that leak to B-root support our 
conclusion that malware still heavily utilizes older systems like Namecoin and 
Emercoin, but has not yet adopted new systems like ENS or 
Unstoppable Domains. We predict that this is due to two 
factors. First, the monetary cost of creating and modifying 
names in ENS and Unstoppable Domains is much higher than in 
naming-specific systems like Namecoin and Emercoin. Second, 
defenders have apparently not yet been able to exert enough 
pressure on Namecoin and Emercoin to make these systems 
unattractive to malware operators, because we still see high 
malware usage of those systems. We hope that the findings in 
this work will aid defenders in exerting more effective 
pressure against malware operators.
