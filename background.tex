\section{Background}
\label{sec:background}

From a malware author's perspective, an ideal naming system for C2 
addresses should have two properties: it should be difficult to censor specific 
individual names, and it should also be difficult to take down the system as a 
whole. For specific names to be difficult to censor, there must be no central 
authority that has the ability or inclination to take down an individual 
record. For a system as a whole to be takedown-resistant, the system should be 
valuable enough to licit actors that authorities cannot block access to it 
entirely without causing significant collateral damage to benign users.

To some extent, a trade-off exists between these features. For example,  
protocols like Tor provide high resistance to the 
censorship of specific names but also stand out in network traffic because they 
are rarely used. This gives defenders the option to block all Tor requests 
entirely, under the assumption that very few licit activities will be 
disrupted. On the other side of 
the spectrum, malware has repurposed ubiquitous, benign systems 
such as social media to store C2 addresses, because defenders do not 
want to impose blanket bans on applications accessing social 
media URLs. However, social media companies such as Facebook and 
Twitter have the capability, motivation, and legal obligation to enforce 
legal takedown requests on individual posts. Malware authors are incentivized 
to find naming systems that are neither vulnerable to censorship of individual 
records nor likely to be blocked or taken down entirely.

\subsection{Tradeoffs of DNS-based C2 names}

Traditional DNS is widely used by malware as the naming system for C2 
servers, but it fulfills only one of the two ideal requirements for malware. 
The tradeoffs inherent in DNS have created a system of incentives for malware 
authors and defenders, upon which defenders' current best tools for malware 
takedowns are based. We now describe the current state of defenses and 
interventions against malware that uses DNS to contact its C2 servers.
 
Since nearly every device and application on the Internet 
requires DNS, enterprises and firewalls almost never block DNS entirely. 
However, DNS is subject to a hierarchy of authorities, who may take down abused 
domains either voluntarily or under legal pressure . Specifically, the 
registrar that sold a domain can delete it, sinkhole it, or 
prevent it from being updated or transferred. One of the primary tools that 
defenders use to take down malware networks is legally compelling registrars to 
take down C2 domains.

\randall{Stefan should look at the following two paragraphs to make this more 
abstract and also add something abut how this only works in jurisdictions 
amenable to this kind of coercion.}
The usual process for enacting a legal takedown in the United States works as 
follows~\cite{knight_domain_2015}. Upon identifying a domain that is being used 
as a C2 center, a law enforcement entity may choose to make 
an Acceptable Use Policy (AUP) complaint, which some registrars comply with 
voluntarily. They may also choose to legally compel the registrar to take down 
the domain by 
seeking a court order. The court order commonly prevents the domain from being 
updated or transferred rather than deleting it entirely, because if 
the domain is deleted, it can be re-registered by the malware authors. It also 
specifies whether the domain should continue to resolve, resolve to a new IP 
address specified by the order, or stop resolving altogether. 

Court orders may be obtained by civil parties as well. The most common method 
is for a company to 
apply for a temporary restraining order (TRO), which orders the perpetrators 
of the offending 
activity to cease and desist and requires any intermediaries that provide 
services to the 
perpetrators to cease providing those services~\cite{kesari_deterring_2017}. 
The latter requirement 
is what allows companies to require registrars to ``sinkhole'' C2 
domains. 

Legal domain takedown orders are a critical tool for defenders to disable 
botnets. For example, 
Microsoft obtained various court orders allowing it to sinkhole the C2 servers 
of the botnet 
Citadel in 2013. Microsoft successfully argued that Citadel caused the Windows 
operating system to 
behave maliciously and frustrate users while still bearing Microsoft 
trademarks, as well as forcing 
Microsoft to spend money on security features to combat 
it~\cite{lerner_microsoft_2014}. The 
Coreflood, Kelihos, and Rustock botnets were each disabled using legal 
takedown orders obtained by 
Microsoft, Pfizer (which claimed to suffer reputational harm because Rustock 
sent spam emails for 
counterfeit pharmaceuticals), and the Department of 
Justice~\cite{kesari_deterring_2017}. 

To complicate the issue, some malware authors have responded to the threat of 
domain 
takedowns by developing Domain Generation Algorithms (DGAs). When an infected 
host uses a DGA, it can randomly generate a large number 
of domains that its C2 server might be found at. DGAs usually use the current 
date as an input, which allows them to be kept in sync and changed on whatever 
time scale is convenient for the malware operators. Only a few of these domains 
will be registered at a time, to prevent defenders from preemptively taking 
them down. If a  
domain that is currently used to contact the C2 server is seized, the malware 
operators simply register a new one. Upon failing to reach the old C2 domain,  
every infected host will begin trying to resolve the rest of the names 
generated by the DGA until they discover the new, working C2 domain. Defenders 
must register all of the domains that the DGA can generate to evade this threat 
entirely~\cite{antonakakis_throw-away_2012}. This intervention is costly in 
terms of money, but has been used successfully in the 
past~\cite{citation_needed}.

Legal takedowns, whether of a single name or all the names a DGA can generate, 
are possible for two reasons. First, some centralized entity controls the C2 
servers' domain names, and thus had the capability to take them down when 
legally compelled to. Second, this centralized entity was located in a legal 
jurisdiction that was amenable to such efforts. Malware 
authors are thus incentivized to find naming systems that are not vulnerable to 
legal takedowns, either because the authority that can control the names is not 
in a takedown-amenable jurisdiction, or because no such authority exists.  

%Botnets 
%often cycle through domain names and IP addresses for their C2 servers 
%quickly, to replace 
%blocklisted domains and IPs or prevent defenders from seizing 
%them~\cite{nadji_beheading_2013}. 
%Malware authors commonly use two strategies to cycle through records: fast 
%flux and AGDs (Algorithmically Generated Domains). Fast flux 
%is the practice of storing multiple records with low TTLs at a domain, and 
%changing the records as 
%soon as the TTL expires. This allows a domain to resolve to a different group 
%of IP addresses every 
%few minutes~\cite{holz_measuring_2008}. The IP addresses in question belong 
%to a pool of 
%compromised machines that can then route requests to the true C2 server. Fast 
%flux increases the 
%number of machines defenders must seize or neutralize, because if any of the 
%compromised machines 
%are still routing traffic, the system still works. However, fast flux is 
%still vulnerable to domain 
%takedowns. Malware authors address this weakness by cycling through domains as 
%well as IP addresses.

\subsection{Blockchain-based domain names}

Blockchain-based naming systems present a potential threat 
because they claim to be immune to takedowns. This supposed immunity stems 
from several factors. 
First, no central authority 
controls blockchain domains in the same way that registrars 
control traditional DNS names. Unlike a DNS record, it is not generally 
possible 
to modify or delete a record on a blockchain without controlling the record's 
private key. Once a domain has been registered, its ownership is passed to the 
purchaser, after which 
point even the company that 
sold it cannot modify it.\footnote{This is true except in the case where the 
seller provides a domain parking service - we discuss this case in 	
Section~\ref{sec:unstoppable_overview}.}
Second, the machines running a blockchain are often distributed across so many 
countries and jurisdictions that seizing or taking down the entire system is 
prohibitively impractical. 
Third, once created, name records are stored immutably on the 
blockchain for as long as that blockchain exists, even if the owner later 
modifies or deletes them. With sufficient resources, and assuming all data is 
stored on-chain, it is possible to reconstruct the value of a 
blockchain-based naming record at any point in time, by parsing the 
transactions that modified the record's state up to that point.

These censorship-resistant properties are generally true under the assumption 
that an adversary has not found a way to compromise the entire blockchain, e.g. 
by performing a 51\% attack. Such 
attacks do exist, but are out of scope for this work. We focus instead on the 
common case where the blockchain underlying a naming system is not compromised, 
in which case the naming system as a whole is highly resistant to takedown 
efforts. Furthermore, some blockchains have become popular enough that even 
blocking access to them at a network level would cause collateral damage to 
licit users. Blockchains such as Bitcoin and Ethereum have recently 
skyrocketed in popularity as investors became interested in cryptocurrency as 
an asset class. As far as we are aware, cryptocurrencies and the blockchains 
they rely on are the first examples of strongly censorship-resistant systems 
that have gained a substantial community of legitimate users around the world.

Blockchain-based naming systems therefore provide both desirable properties of 
naming systems for C2 servers: it is difficult to take down the whole system as 
well as to take over individual records. Unfortunately, some malware is already 
aware of these advantages. BazarLoader uses Emercoin 
to record the domains of its C2 servers~\cite{brandt_bazarloader_2021}. 
Namecoin is used by the Necurs botnet~\cite{dgas_of_necurs}, the 
Chthonic banking trojan~\cite{malware_traffic_analysis_2016}, Smoke 
Loader/Dofoil, Backdoor.Teamviewer, Shifu, and TinyNuke~\cite{abusech_2017, 
mackie_cryptodns_2018}. Cerber ransomware has 
even used blockchain wallet addresses as names for its C2 
servers~\cite{pletinckx_malware_2018}.

\randall{is this the thing Aaron says not to do where I shouldn't put 
transitions at the end of sections?}
However, we find that blockchain naming systems are not as invincible as they 
claim. While interventions that work on DNS domains may not work on blockchain 
names, blockchain naming systems have two different 
weaknesses that DNS does not have: the difficulty of accessing the blockchain,  
and the challenges of modifying records. In the remainder of this work, we 
detail the specific challenges posed by the blockchain DNS ecosystem to malware 
authors and defenders, as well as the pieces of the ecosystem where 
effective interventions might be staged.

%Blockchain domains are not accessible through traditional DNS. Instead, 
%malware 
%must find a way to contact a member of the blockchain to resolve its chosen 
%domain. Malware authors 
%can implement this in two ways: they can either use a centralized proxy with a 
%known domain to gain 
%access to the blockchain, or they can act as first-class participants of the 
%blockchain themselves, 
%and participate in the peer-to-peer discovery protocol. Both approaches 
%present opportunities for 
%defenders to intervene. 
%
%Modifying records on blockchains also poses new challenges for both defenders 
%and malware authors. 
%Modifications are usually more expensive (in terms of money) than in DNS, 
%making fast flux or DGA 
%impractical on some chains. Similarly, if a defender wishes to register 
%all domains that a DGA 
%might produce, this becomes expensive on some chains.

