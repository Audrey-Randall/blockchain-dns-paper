\section{Background}

From a malware author's perspective, an ideal naming system for C2 
addresses must be uncensorable at both the request level and the system level. 
To be 
uncensorable at the request level, there should be no central authority that 
has the ability to enforce a legal takedown notice for an individual record. To 
be uncensorable at the system level, the system should be valuable enough to 
licit actors that authorities cannot block access to it entirely without 
causing significant collateral damage to benign users.

To some extent, a trade-off exists between these features. On 
one side of the 
spectrum, protocols like Tor provide high resistance to request-level 
censorship, but they stand out in network traffic, allowing systems like 
IDSes to detect the 
malware's presence. For example, enterprise networks often block Tor 
entirely under the assumption 
that none of their employees will use it for any legitimate purpose.
%but do not have enough licit traffic to prevent authorities from 
%blocking access to the system entirely. \randall{Cite that paper that   
	%said when you take away the malware, what's left is 80\% CSAM, and find 
	%other 
	%citations that say defenders block Tor.}
On the other side of 
the scale, malware has repurposed ubiquitous, benign systems 
such as social media to store C2 addresses. Defenders do not 
want to impose blanket bans on applications accessing social 
media URLs, but social media companies such as Facebook and 
Twitter have the capability, motivation, and legal obligation to enforce 
legal takedown requests on individual posts. Malware authors want to use 
systems that are balanced 
between being difficult to block at the record level and difficult to detect 
and block at the 
system level.

\subsection{DNS-based domain names}

Traditional DNS has high system-level resistance to censorship but low 
request-level resistance. 
Since nearly every device and application on the Internet requires DNS, 
enterprises and firewalls 
almost never block it entirely. However, while DNS is a decentralized 
system in terms of how its 
components are replicated across the globe, it is not decentralized in terms 
of authority. The 
registrar that sold a domain can be compelled to ``delete'' that domain or 
prevent it from being 
updated or transferred. 

The usual process for enacting a legal takedown in the United States works as 
follows~\cite{knight_domain_2015}. Upon 
identifying a domain that is being used as a C2 center, a law enforcement 
entity may choose to make 
an Acceptable Use Policy (AUP) complaint or may immediately seek a court order 
compelling the 
registrar to take down the domain. Some registrars cooperate with AUC 
complaints without legal 
intervention, but others do not. If the registrar does not respond to the 
AUC complaint and take 
down the domain, law enforcement may move on to using a court order. The 
court order commonly 
prevents the domain from being updated or transferred rather than deleting it 
entirely, because if 
the domain is deleted, it can be re-registered by the malware authors. The 
court order also 
specifies whether the domain should continue to resolve, resolve to a new IP 
address specified by 
the order, or stop resolving altogether. 

Court orders may be obtained by civil parties as well. The most common method 
is for a company to 
apply for a temporary restraining order (TRO), which orders the perpetrators 
of the offending 
activity to cease and desist and requires any intermediaries that provide 
services to the 
perpetrators to cease providing those services~\cite{kesari_deterring_2017}. 
The latter requirement 
is what allows companies to require registrars to ``sinkhole'' C2 
domains. 

Legal domain takedown orders are a critical tool for defenders to disable 
botnets. For example, 
Microsoft obtained various court orders allowing it to sinkhole the C2 servers 
of the botnet 
Citadel in 2013. Microsoft successfully argued that Citadel caused the Windows 
operating system to 
behave maliciously and frustrate users while still bearing Microsoft 
trademarks, as well as forcing 
Microsoft to spend money on security features to combat 
it~\cite{lerner_microsoft_2014}. The 
Coreflood, Kelihos, and Rustock botnets were each disabled using legal 
takedown orders obtained by 
Microsoft, Pfizer (which claimed to suffer reputational harm because Rustock 
sent spam emails for 
counterfeit pharmaceuticals), and the Department of 
Justice~\cite{kesari_deterring_2017}. These 
takedowns are possible because in each case, some centralized entity had 
control of the C2 servers' 
domain names, and thus had the capability to take them down when legally 
compelled to. Malware 
authors are thus incentivized to find naming systems that are not vulnerable to 
legal takedowns.  

\subsection{Blockchain-based domain names}

Blockchain-based naming systems present a potential threat 
because they claim to be immune to takedowns. No central authority 
controls blockchain domains in the same way that registrars 
control traditional DNS names. In general, it is not possible to modify or 
delete a record on a blockchain without controlling the record's private 
key. Once a domain has 
been registered, its ownership is passed to the purchaser, after which 
point even the company that 
sold it cannot modify it. The record is also stored immutably on the 
blockchain for as long as the 
chain exists, even if the owner later modifies or deletes it, because 
historical recods are part of 
the blockchain and cannot be erased. Blockchain-based naming systems 
therefore have high resistance 
to request-level censorship.

Furthermore, some blockchains are popular enough to be challenging to block 
without 
harming licit users. Blockchains such as Bitcoin and Ethereum have recently 
skyrocketed in popularity as investors became interested in cryptocurrency as 
an asset class. Because these blockchains are no longer niche systems used 
only by a few 
technologically-savvy users, blocking access 
to them entirely may be difficult without angering a large number of 
legitimate users. As far as we are aware, cryptocurrencies and the blockchains 
they rely on are the first examples of strongly censorship-resistant systems 
that have gained a substantial community of legitimate users around the world.

Blockchain is therefore an attractive option for malware authors to use as a 
naming 
layer for C2 server addresses. In fact, some malware is already exploiting 
these tools.
BazarLoader uses the Emercoin ``.bazar'' TLD to record the domains of its C2 
servers~\cite{brandt_bazarloader_2021}. The Namecoin TLD 
``.bit'' is used by the
Necurs botnet~\cite{dgas_of_necurs}, the Chthonic banking 
trojan~\cite{malware_traffic_analysis_2016}, Smoke Loader/Dofoil, 
Backdoor.Teamviewer, Shifu, 
and TinyNuke~\cite{abusech_2017, mackie_cryptodns_2018}. Cerber ransomware 
has 
even used blockchain wallet addresses as names for its C2 
servers~\cite{pletinckx_malware_2018}.

However, we find that blockchain domains are not as invincible as they 
claim. While sinkholing 
blockchain domains through their registrars is not possible, blockchain 
domains have two different 
weaknesses that DNS does not have: accessing the system and modifying records.

Blockchain domains are not accessible through traditional DNS. Instead, 
malware 
must find a way to contact a member of the blockchain to resolve its chosen 
domain. Malware authors 
can implement this in two ways: they can either use a centralized proxy with a 
known domain to gain 
access to the blockchain, or they can act as first-class participants of the 
blockchain themselves, 
and participate in the peer-to-peer discovery protocol. Both approaches 
present opportunities for 
defenders to intervene. 

Modifying records on blockchains also poses new challenges for both defenders 
and malware authors. 
Modifications are usually more expensive (in terms of money) than in DNS, 
making fast flux or DGA 
impractical on some chains. Similarly, if a defender wishes to register 
all domains that a DGA 
might produce, this becomes expensive on some chains.

In the remainder of this work, we detail the specific challenges posed by the 
blockchain DNS 
ecosystem to malware authors and defenders, as well as the pieces of the 
ecosystem where 
effective interventions might be staged.