\section{Background}
\label{sec:background}

From a malware author's perspective, an ideal naming system for C2
addresses should have two properties: it should be difficult to censor
specific individual names, and it should also be costly to take down
or block the system as a whole.  Typically, this first property,
per-name takedown-resistance, is directly tied to the existence of a
central authority with the ability to take down an individual record.
System-wide takedown-resistance, on the other hand, is commonly a
byproduct of a systems overall utility to legitimate users; potential
collateral damage to benign users can create strong pressures against
such takedowns.

To some extent, a trade-off exists between these features. For
example, protocols such as Tor provide high resistance to the
censorship of specific names, but because Tor traffic is both easily
identified and represents a small volume of users, some defenders will
simply block Tor altogether (i.e., under the assumption that any
damage on benign users will be minor).  On the other side of the
spectrum, malware has repurposed ubiquitous systems such as social
media to store C2 addresses, because such traffic does not stand out
and blanket bans on social media URLs are practically untenable.
However, social media companies such as Facebook and Twitter have the
capability, motivation, and (at times) legal obligation to remove
abusive posts used to coordinate C2 activity.  Thus, malware authors
are incentivized to find naming systems that are neither vulnerable to
censorship of individual records nor likely to be blocked or taken
down entirely.

\subsection{Tradeoffs of DNS-based C2 names}

In part due to its ubiquity, DNS has been one of the most widely used
naming systems for malware C2 servers.  Because of its central role in
the Internet's function, DNS naturally satisfies the system-wide
takedown-resistance requirement; completely blocking DNS would be
unthinkable for any modern enterprise or ISP.  However, DNS is subject
to a hierarchy of defined authorities, each of which may disable
individual domain names, either voluntarily or in response to legal
compulsion.  For example, the registrar that sold a domain can delete
it, or sinkhole it (i.e., divert its traffic) and prevent it from
being subsequently updated or transferred.  Similarly, the registry
responsible for a domain name's top-level domain (TLD) can also take
such actions.

Key to all these actions is that there is a singular legal entity with
the capability to intervene.  In some cases, they may do so
voluntarily (e.g., such as when a registrar is notified of a violation
of their terms of service.\footnote{Registrars and hosting providers
  all typically specify a ``Acceptable Use Policy'' (AUP) in their
  contracts with third-parties.  The details of these AUPs vary
  between providers, but it is common that they prohibit activity that
  is criminal, that violates intellectual property interests, or is
  highly disruptive.}) but they may also be compelled to take action
by a court order.  For example, US law enforcement may obtain a
warrant to seize control of a domain name (subject to a showing of
probable cause that it is involved in a criminal act), so long as the
controlling registrar or registry is within US
jurisdiction.\footnote{This process is described in considerable
  detail by Knight~\cite{knight_domain_2015}.}  Typically such names
are not then deleted (which would allow the malware author to
re-register them), but instead the registrar (or registry) is
compelled to redirect traffic to a benign site (aka a sinkhole) and
block any attempts to change or transfer the domain by the registrant.
Civil litigants can obtain similar effects via Temporary Restraining
Orders (TRO) typically based on a showing that their intellectual
property rights are being violated~\cite{kesari_deterring_2017}.
Microsoft, in particular, has made innovative use of trademark
protection laws, such as the Lanham Act, to drive expansive
private-sector botnet takedown efforts~\cite{lerner_microsoft_2014}.

Today such legal takedowns are a critical tool for defenders to
disable botnets.  However, the effectiveness of this tool has led
malware authors to respond by developing Domain Generation Algorithms
(DGAs) which minimize the impact of any given takedown. When an
infected host uses a DGA, it can randomly generate a large number of
domains that its C2 server might be found at. DGAs usually use the
current date as an input, which allows them to be kept in sync and
changed on whatever time scale is convenient for the malware
operators. Only a few of these domains will be registered at a time,
to prevent defenders from preemptively taking them down. If a domain
that is currently used to contact the C2 server is seized, the malware
operators simply register a new one. Upon failing to reach the old C2
domain, every infected host will begin trying to resolve the rest of
the names generated by the DGA until they discover the new, working C2
domain. Thus, to truly takedown such a C2, defenders must register all
of the domains (or otherwise prevent them from being registered) that
the DGA can generate~\cite{antonakakis_throw-away_2012}. This
intervention is costly in effort, but has been used successfully in
the past~\cite{confickersomething}.


To summarize, DNS-based C2 represent the status quo for modern
botnets, but are vulnerable to concerted legal takedown efforts.  What
makes these takedowns possible is the existence of singular entities
with the capability to unilaterally assert control over individual
names \emph{and} that those same entities are subject to legal
jurisdictions available to defenders.  Malware authors are thus
incentivized to find naming systems that are not vulnerable to such
efforts, either because the authority that can control the names is
not in a takedown-amenable jurisdiction, or because no such authority
exists.

%Botnets 
%often cycle through domain names and IP addresses for their C2 servers 
%quickly, to replace 
%blocklisted domains and IPs or prevent defenders from seizing 
%them~\cite{nadji_beheading_2013}. 
%Malware authors commonly use two strategies to cycle through records: fast 
%flux and AGDs (Algorithmically Generated Domains). Fast flux 
%is the practice of storing multiple records with low TTLs at a domain, and 
%changing the records as 
%soon as the TTL expires. This allows a domain to resolve to a different group 
%of IP addresses every 
%few minutes~\cite{holz_measuring_2008}. The IP addresses in question belong 
%to a pool of 
%compromised machines that can then route requests to the true C2 server. Fast 
%flux increases the 
%number of machines defenders must seize or neutralize, because if any of the 
%compromised machines 
%are still routing traffic, the system still works. However, fast flux is 
%still vulnerable to domain 
%takedowns. Malware authors address this weakness by cycling through domains as 
%well as IP addresses.

\subsection{Blockchain-based domain names}

Blockchain-based naming systems present a potential threat 
because they claim to be immune to takedowns. This supposed immunity stems 
from several factors. 
First, no central authority 
controls blockchain domains in the same way that registrars 
control traditional DNS names. Unlike a DNS record, it is not generally 
possible 
to modify or delete a record on a blockchain without controlling the record's 
private key. Once a domain has been registered, its ownership is passed to the 
purchaser, after which 
point even the company that 
sold it cannot modify it.\footnote{This is true except in the case where the 
seller provides a domain parking service - we discuss this case in 	
Section~\ref{sec:interventions_at_name}.}
Second, the machines running a blockchain are often distributed across so many 
countries and jurisdictions that seizing or taking down the entire system is 
prohibitively impractical. 
Third, once created, name records are stored immutably on the 
blockchain for as long as that blockchain exists, even if the owner later 
modifies or deletes them. With sufficient resources, and assuming all data is 
stored on-chain, it is possible to reconstruct the value of a 
blockchain-based naming record at any point in time, by parsing the 
transactions that modified the record's state up to that point.

These censorship-resistant properties are generally true under the assumption 
that an adversary has not found a way to compromise the entire blockchain, e.g. 
by performing a 51\% attack. Such 
attacks do exist, but are out of scope for this work. We focus instead on the 
common case where the blockchain underlying a naming system is not compromised, 
in which case the naming system as a whole is highly resistant to takedown 
efforts. Furthermore, some blockchains have become popular enough that even 
blocking access to them at a network level would cause collateral damage to 
licit users. Blockchains such as Bitcoin and Ethereum have recently 
skyrocketed in popularity as investors became interested in cryptocurrency as 
an asset class. As far as we are aware, cryptocurrencies and the blockchains 
they rely on are the first examples of strongly censorship-resistant systems 
that have gained a substantial community of legitimate users around the world.

Blockchain-based naming systems therefore provide both desirable properties of 
naming systems for C2 servers: it is difficult to take down the whole system as 
well as to take over individual records. Unfortunately, some malware is already 
aware of these advantages. BazarLoader uses Emercoin 
to record the domains of its C2 servers~\cite{brandt_bazarloader_2021}. 
Namecoin is used by the Necurs botnet~\cite{dgas_of_necurs}, the 
Chthonic banking trojan~\cite{malware_traffic_analysis_2016}, Smoke 
Loader/Dofoil, Backdoor.Teamviewer, Shifu, and TinyNuke~\cite{abusech_2017, 
mackie_cryptodns_2018}. Cerber ransomware has 
even used blockchain wallet addresses as names for its C2 
servers~\cite{pletinckx_malware_2018}.

\randall{is this the thing Aaron says not to do where I shouldn't put 
transitions at the end of sections?}
However, we find that blockchain naming systems are not as invincible as they 
claim. While interventions that work on DNS domains may not work on blockchain 
names, blockchain naming systems have two different 
weaknesses that DNS does not have: the difficulty of accessing the blockchain,  
and the challenges of modifying records. In the remainder of this work, we 
detail the specific challenges posed by the blockchain DNS ecosystem to malware 
authors and defenders, as well as the pieces of the ecosystem where 
effective interventions might be staged.

%Blockchain domains are not accessible through traditional DNS. Instead, 
%malware 
%must find a way to contact a member of the blockchain to resolve its chosen 
%domain. Malware authors 
%can implement this in two ways: they can either use a centralized proxy with a 
%known domain to gain 
%access to the blockchain, or they can act as first-class participants of the 
%blockchain themselves, 
%and participate in the peer-to-peer discovery protocol. Both approaches 
%present opportunities for 
%defenders to intervene. 
%
%Modifying records on blockchains also poses new challenges for both defenders 
%and malware authors. 
%Modifications are usually more expensive (in terms of money) than in DNS, 
%making fast flux or DGA 
%impractical on some chains. Similarly, if a defender wishes to register 
%all domains that a DGA 
%might produce, this becomes expensive on some chains.

