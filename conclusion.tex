\section{Conclusion}
\label{sec:conclusion}

While decentralized naming and hosting systems pose challenges, they cannot 
entirely 
eliminate their reliance on systems with centralized authority. Whenever 
malware uses a centralized 
resource to enable its use of decentralized ones, defenders can intervene. 
Defenders cannot serve 
legal takedown orders to a centralized registrar to take 
down a blockchain domain, but they can prevent malware from accessing the 
blockchain in the first 
place, or target the DNS domain or IP address that the blockchain domain 
resolves to. We examined existing blockchain-based naming systems and found 
that naming systems on general purpose blockchains doare not 
currenly attractive to malware because of their high cost. In 
contrast, systems on naming-specific blockchains present an 
ongoing threat, but these systems have 
been well studied. They are also susceptible 
to defenses such as 
blocklisting every IP address stored in the name records, 
blocking the proxies that resolve the names, or blocking 
the system entirely, because so little licit 
content exists on those blockchains. We conclude that for a naming system to be 
truly more dangerous than DNS, it must achieve widespread 
adoption as well as inexpensive transactions and high 
ease-of-use, and no existing naming systems have yet achieved 
all three characteristics. 